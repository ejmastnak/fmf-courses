\documentclass[11pt, a4paper]{article}
\usepackage[T1]{fontenc}
\usepackage{geometry, multicol, adjustbox}
\usepackage{microtype, xspace}
\usepackage[breakable]{tcolorbox}
\usepackage[shortlabels]{enumitem}
\usepackage{amsmath, amssymb, bm}
\usepackage{physics, siunitx}
\usepackage{hyperref}

\newcommand{\correspondsto}{$ \iff $}
\renewcommand{\vec}{\bm}

% Package options
% --------------------------------------------- %
\geometry{margin=3.0cm}
\hypersetup{colorlinks=true, linkcolor=blue, urlcolor=cyan}
\sisetup{separate-uncertainty=true, exponent-product=\cdot, range-units=single}
\setlength{\parindent}{0pt}
\setlist{itemsep=-3.0pt, topsep=3pt}
% \setcounter{section}{-1}
% --------------------------------------------- %

% Header configuration
% --------------------------------------------- %
\usepackage{fancyhdr}
\usepackage{extramarks}
\setlength{\headheight}{13.6pt}
\fancyhf{}
\fancyhead[L]{\leftmark}  % section
% \fancyhead[R]{\firstrightmark}  % subsection
\fancyfoot[C]{\thepage} 
% --------------------------------------------- %

% Custom environments
% --------------------------------------------- %
\newenvironment{course}[3]{
\subsection{#1}%
Link: \href{#2}{Official FMF description}\\%
ECTS: #3%
\vspace{1ex}
\\
{\large \textsc{Contents}}\\[-0.9ex]%
\rule{\textwidth}{0.5pt}
\vspace{-3ex}
}
{}

\newenvironment{chapter}[1]{
\begin{tcolorbox}[title=#1, breakable]
}
{\end{tcolorbox}}
% --------------------------------------------- %

\begin{document}

\pagestyle{empty}  % remove headers from title page

\begin{center}
\thispagestyle{empty}
\null
\vspace{20ex}
{\Huge Physics Curriculum at FMF}
\rule{0.75\textwidth}{0.5pt}
\vspace{2ex}

\begin{minipage}[t]{0.90\textwidth}
    This document describes the curriculum of the undergraduate Physics program at the Faculty of Mathematics and Physics (FMF) at the University of Ljubljana.
    The core curriculum is covered in entirety, but coverage of elective courses is currently limited to those taken by the author, Elijan Mastnak.
    
    \vspace{1ex}
    The document's \LaTeX{} source code is available under a \href{http://creativecommons.org/licenses/by-nc-sa/4.0/}{CC BY-NC-SA 4.0} license at \href{https://github.com/ejmastnak/fmf-courses}{\texttt{github.com/ejmastnak/fmf-courses}}; pull requests to add additional elective courses are very much welcome.

    \vspace{2ex}
    \textbf{\large Navigating the table of contents}\\[0.5ex]
    The table of contents appears on the next page. 
    It is fully hyperlinked for easier document navigation and contains the following five columns:
    \begin{itemize}
        \item The \textbf{Course} column contains the course name exactly as it appears on the official academic transcript issued by the University of Ljubljana;
        each blue course name is a clickable hyperlink to the position in the document at which the course is described.

        \item The \textbf{Clarification} column describes the course content in situations where the official name is ambiguous (e.g. \textit{Mathematics I} covers real analysis).

        \item The \textbf{ETCS} column shows the number of ETCS credits awarded for completing the course.

        \item The \textbf{Year} column shows the year of study in which the course was taken.

        \item The \textbf{Page} column is a clickable hyperlink to the page number where the course is described.

    \end{itemize}

    \vspace{2ex}
    \textbf{\large About the course descriptions}\\[0.5ex]
    Each course's material is divided by chapter and presented in bullet-point form, with the following information included at the start of each course description:
    \begin{itemize}
    
        \item A link to an online, official University of Ljubljana description of each course
        (note that the official descriptions are less complete than those given in this document and may occasionally contain typos or poor English);

        \item the ETCS awarded for completing the course.

    \end{itemize}
\end{minipage}

\vfill

\rule{0.45\linewidth}{0.4pt}\\
Elijan Mastnak\\
Faculty of Mathematics and Physics, University of Ljubljana\\
Last update: \today

\end{center}
\newpage
% --------------------------------------------- %
% END TITLEPAGE

% BEGIN TOC
% --------------------------------------------- %
\section*{Contents}
\pagestyle{empty}

\begin{adjustbox}{center}
    \small
    \begin{tabular}{|l|l|c|c|c|}
    \hline {\rule{0pt}{2.0ex}} \hspace{-7pt}
    \textbf{Course} & \multicolumn{1}{c}{\textbf{Clarification}} & \textbf{ECTS} & \textbf{Year} & \textbf{Page}\\
    \hline
    \hline {\rule{0pt}{2.5ex}} \hspace{-7pt}
    \hyperref[mathematics_1]{Mathematics I} & Real analysis & 10 & 1 & \pageref{mathematics_1} \\
    \hyperref[mathematics_2]{Mathematics II} & Linear algebra \& multivariable calculus & 8 & 1 & \pageref{mathematics_2} \\
    \hyperref[classical_physics]{Classical Physics} & General physics & 20 & 1 & \pageref{classical_physics} \\
    \hyperref[computational_tools_in_physics]{Computational Tools in Physics} & & 3 & 1 & \pageref{computational_tools_in_physics} \\
    \hyperref[proseminar_a]{Proseminar A} & Mathematical methods for classical physics & 3 & 1 & \pageref{proseminar_a} \\
    \hyperref[physics_laboratory_1]{Physics Laboratory I} & & 4 & 1 & \pageref{physics_laboratory_1} \\
    \hyperref[physics_laboratory_2]{Physics Laboratory II} & & 4 & 1 & \pageref{physics_laboratory_2} \\
    \hyperref[computing_laboratory]{Computing Laboratory} & Introduction to programming in \texttt{C} & 3 & 1 & \pageref{computing_laboratory} \\
    \hyperref[chemistry_1]{Chemistry I} & Inorganic chemistry & 3 & 1 & \pageref{chemistry_1} \\
    \hyperref[how_things_work]{How Things Work} & & 3 & 1 & \pageref{how_things_work} \\
    \hline
    \hline

    % Year 2
    \textbf{Course} & \multicolumn{1}{c}{\textbf{Clarification}} & \textbf{ECTS} & \textbf{Year} & \textbf{Page}\\
    \hline
    \hline {\rule{0pt}{2.5ex}} \hspace{-7pt}
    \hyperref[mathematics_3]{Mathematics III} & Vector calculus, LDEs, Hilbert spaces & 8 & 2 & \pageref{mathematics_3} \\
    \hyperref[mathematics_4]{Mathematics IV} & Complex analysis, Fourier analysis, PDEs & 6 & 2 & \pageref{mathematics_4} \\
    \hyperref[modern_physics_1]{Modern Physics I} & Special relativity \& quantum mechanics & 8 & 2 & \pageref{modern_physics_1} \\
    \hyperref[modern_physics_2]{Modern Physics II} & Solid state, nuclear, \& particle physics & 5 & 2 & \pageref{modern_physics_2} \\
    \hyperref[classical_mechanics]{Classical Mechanics} & & 5 & 2 & \pageref{classical_mechanics} \\
    \hyperref[statistical_thermodynamics]{Statistical Thermodynamics} & & 5 & 2 & \pageref{statistical_thermodynamics} \\
    \hyperref[physics_laboratory_3]{Physics Laboratory III} & & 4 & 2 & \pageref{physics_laboratory_3} \\
    \hyperref[physics_laboratory_4]{Physics Laboratory IV} & & 4 & 2 & \pageref{physics_laboratory_4} \\
    \hyperref[probability_in_physics]{Probability in Physics} & Probability and statistics & 3 & 2 & \pageref{probability_in_physics} \\
    \hyperref[numerical_methods]{Numerical Methods} & & 3 & 2 & \pageref{numerical_methods} \\
    \hyperref[electronics_1]{Electronics I} & Analog electronics & 3 & 2 & \pageref{electronics_1} \\
    \hyperref[electronics_2]{Electronics II} & Digital electronics and control theory & 3 & 2 & \pageref{electronics_2} \\
    \hyperref[electronics_laboratory]{Electronics Laboratory} & & 3 & 2 & \pageref{electronics_laboratory} \\
    \hline
    \hline

    % Year 3
    \textbf{Course} & \multicolumn{1}{c}{\textbf{Clarification}} & \textbf{ECTS} & \textbf{Year} & \textbf{Page}\\
    \hline
    \hline {\rule{0pt}{2.5ex}} \hspace{-7pt}
    \hyperref[electromagnetic_fields]{Electromagnetic Fields} & & 7 & 3 & \pageref{electromagnetic_fields} \\
    \hyperref[quantum_mechanics]{Quantum Mechanics} & & 7 & 3 & \pageref{quantum_mechanics} \\
    \hyperref[optics]{Optics} & & 5 & 3 & \pageref{optics} \\
    \hyperref[mechanics_of_continuous_media]{Mechanics of Continuous Media} & & 5 & 3 & \pageref{mechanics_of_continuous_media} \\
    \hyperref[solid_state_physics]{Solid State Physics} & & 7 & 3 & \pageref{solid_state_physics} \\
    \hyperref[mathematical_physics_lab]{Mathematical Physics Lab} & Computational mathematical physics & 6 & 3 & \pageref{mathematical_physics_lab} \\
    \hyperref[physics_laboratory_5]{Physics Laboratory V} & & 7 & 3 & \pageref{physics_laboratory_5} \\
    \hyperref[industrial_physics]{Industrial Physics} & & 3 & 3 & \pageref{industrial_physics} \\
    \hyperref[measurement_of_ionizing_radiation]{Msmnt. of Ionizing Radiation} & & 5 & 3 & \pageref{measurement_of_ionizing_radiation} \\
    \hyperref[collecting_and_processing_data]{Collecting and Processing Data} & Data acquisition \& DSP & 3 & 3 & \pageref{collecting_and_processing_data} \\
    \hyperref[measurement_techniques]{Measurement Techniques} & Instrumentation \& data processing & 7 & 3 & \pageref{measurement_techniques} \\
    \hyperref[seminar]{Seminar} & & 3 & 3 & \pageref{seminar} \\
    \hline
\end{tabular}
\end{adjustbox}

% --------------------------------------------- %
% END TOC

\newpage
\pagestyle{fancy}
\section{Year 1}

\begin{course}{Mathematics I}{https://www.fmf.uni-lj.si/en/study-physics/programmes/1fiz/2020/7000777/courses/520/}{10}
    \label{mathematics_1}
    
    \begin{chapter}{Sets and functions}
        \begin{itemize}
        
            \item Fundamentals of set theory: operations on sets (intersection, difference, union, Cartesian product); complement of a set; indexed families of sets; de Morgan's laws; relations and equivalence relations; set cardinality.

            \item Set-based formulation of functions: definition of a function as a mapping between sets; the domain, codomain, and graph of a function; image of a function and subset of a function's domain; restriction of a function to a subset and inclusion of a function into a superset; injective, surjective, and bijective functions; composition of functions; the inverse and left and right inverse of a function.
        
        \end{itemize}
    \end{chapter}

    \begin{chapter}{Numbers}
        \begin{itemize}
        
            \item Natural numbers: the Peano axioms for the natural numbers; formulation of addition and multiplication using the Peano axioms; mathematical induction. 

            \item Real numbers: axioms defining the real numbers; axiomatic definition of real addition and multiplication; Dedekind's axiom; definition of a bounded set; the supremum, infimum, maximum, and minimum of a set; decimal representation of real numbers; definition and existence of the real square root; intervals and neighborhoods, open and closed intervals; absolute value.

            \item Complex numbers: axioms defining the complex numbers; axiomatic definitions of complex addition and multiplication; component representation of complex number; the absolute value and complex conjugate; polar representation of complex numbers in the plane; complex disks and neighborhoods, open and closed disks; De Moivre's formula and roots of unity.
        
        \end{itemize}
    \end{chapter}

    \begin{chapter}{Geometry of the Euclidean space $ \mathbb{R}^{3} $}
        \begin{itemize}
        
            \item Definition of $ \mathbb{R}^{3} $; the operations of vector addition and scalar multiplication; definition and properties of vector spaces.

            \item Cartesian unit vectors, coordinates, and the standard basis for $ \mathbb{R}^{3} $.

            \item The scalar product: definition and properties of the scalar product; the length of a vector and the distance between two vectors; the Cauchy-Schwartz and triangle inequalities; orthogonal projection onto a subspace of $ \mathbb{R}^{3} $.

            \item Definition and properties of the cross product and the scalar-valued and vector-valued triple product; applications to the area of parallelograms and volumes of parallelepipeds; linear independent and orthogonal vectors.
        
            \item Planes and lines in $ \mathbb{R}^{3} $: equation and parameterization of a plane; the normal vector to a plane and the standard equation of a plane; equation and parameterization of a line; the standard equation of a plane; the distances between planes, points, and lines; angles between lines and planes.

        \end{itemize}
    \end{chapter}

    \begin{chapter}{Sequences of numbers}
        \begin{itemize}
        
            \item Foundational concepts: definition of a sequence of numbers as a function from $ \mathbb{N} $ to $ \mathbb{C} $; bounded sequences; supremum and infimum of a sequence; (strictly) increasing, decreasing, and monotone sequences.

            \item Definition and theory of cluster points of sequences; the Bolzano-Weierstrass theorem;
            definition and theory of the limit of a sequence; computation of limits.

            \item Convergence: definition and theory of convergent and divergent sequences; tests for convergence;
            subsequences: definition, properties, and theory of subsequences;
            Cauchy sequences and the Cauchy criterion for the convergence of a sequence.

            \item Arithmetic operations on sequences; sequence definition of Euler's number $ e $.
        
        \end{itemize}
    \end{chapter}

    \begin{chapter}{Series of numbers}
        \begin{itemize}
        
            \item Foundational concepts: definition of a series of numbers and the sequence of partial sums.

            \item Definition and theory of convergence of series; formulae for the summation of simple series; Cauchy convergence criterion for series; absolute convergence.

            \item Theory and application of series convergence tests: comparison test, root test, quotient test, Raabe's test, the Leibniz test.
        
        \end{itemize}
    \end{chapter}

    \begin{chapter}{Functions of a single variable}
        \begin{itemize}
        
            \item Foundational concepts: definition of scalar function; (strictly) increasing, decreasing, and monotone functions; bounded functions and the supremum, infimum, minimum, and maximum of a function; even and odd functions.

            \item Operations with functions: the sum, product, and quotient of a function; function composition and the review of the inverse of a function.

            \item Limit of a function: cluster points and isolated points of subsets of $ \mathbb{R} $; definition of the limit of a function; left and right limits; asymptotes and limits at infinity; properties and computation of limits.

            \item Continuity of functions: definition of continuity at a point and on the entire domain; theory and properties of continuous real functions; definition and theory of uniform continuity.

            \item Examples of common functions:  the identify function, constant functions, linear, quadratic, and general polynomial functions; the exponential, logarithmic, and power functions; hyperbolic functions; the trigonometric and cyclometric functions.

        \end{itemize}
    \end{chapter}

    \begin{chapter}{Differential calculus}
        \begin{itemize}
        
            \item Foundational concepts: definition of differentiability at a point and on a function's domain and the definition of a function's derivative; relationship of differentiability and continuity; higher derivatives and smooth functions.

            \item Differentiation rules: derivatives of sums of functions and scaled functions; the product, quotient, and chain rules;
            derivatives of common functions: polynomials, rational functions, power functions, the square root, the logarithm and exponential functions; sinusoidal functions.

            \item Extrema: definitions of local and global maxima and minima;
            Rolle's theorem and Lagrange's and Cauchy's mean value theorems;
            necessary and sufficient conditions for extrema; application of the derivative to the computation of extrema.

            \item Definitions and theory of convexity, concavity, and points of inflection; tests for convexity and concavity; application of the derivative to curve sketching.

            \item Applications: L'Hopital's rule for the computation of limits and Newton's method for computing the zeros of functions.

        \end{itemize}
    \end{chapter}

    \begin{chapter}{Integral calculus}
        \begin{itemize}
            \item Definition and properties of the indefinite integral; computation of the indefinite integrals of polynomial, power, rational, trigonometric, exponential, and certain irrational functions.

            \item Techniques for indefinite integration: integration by parts, change of variables; partial fraction decomposition; integration using ansatzes; integration of trigonometric functions.

            \item Darboux integration: partitions, upper and lower sums; upper and lower integrals; definition of Darboux integrability and the Darboux integral of a function; theory of Darboux-integrable functions.

            \item Riemann integration: definition of Riemann integrability and the Riemann integral of a function; relationship of Riemann and Darboux integrability; theory of Riemann-integrable functions.

            \item The average value of a function and the first and second fundamental theorems of calculus.

            \item Numerical integration: the trapezoidal and Simpson's rule

            \item Improper integrals: definition, convergence, and practical computation of integrals at infinity and integrals crossing asymptotes and discontinuities.

            \item Applications of integration: area under a curve, surface area and volume of a surface of revolution; arc length of a curve.

            \item Curves in three-dimensional space: parameterization of curves; arc length; curvature and torsion; the tangent, normal, and binormal vectors and the Frenet–Serret formulae.

        \end{itemize}
    \end{chapter}

    \begin{chapter}{Taylor and power series}
        \begin{itemize}
        
            \item Foundational concepts: the $ n $-th degree Taylor polynomial of function; Taylor's formula and the Lagrange and Cauchy form of the remainder.

            \item Definition and properties of the Taylor series of a function.

            \item Taylor series for the exponential, sine, cosine, and logarithm functions.

            \item Definition and properties of complex power series; definition of the exponential function on the complex numbers.

            \item Convergence of power series: convergence and absolute convergence; radius of convergence; Cauchy product of power series.

            \item Series of functions: uniform and point-wise convergence of function sequences; continuity and differentiability of function series; differentiation and integration of function series.
        
        \end{itemize}
    \end{chapter}

    \begin{chapter}{Higher-dimensional Euclidean space}
        \begin{itemize}
        
            \item Foundational concepts: generalization of intervals, openness, and boundedness to $ \mathbb{R}^{N} $; distance and orthogonality in $ \mathbb{R}^{N} $; path-connected and compact sets in $ \mathbb{R}^{N} $.

            \item Vector-valued functions of multiple variables: domain and graph of vector-valued functions; limits and continuity of vector valued functions

            \item Differentiation of vector-valued functions: partial differentiability and partial derivatives; the tangent hyperplane; total differentiability; the gradient operator and directional derivatives; higher-order partial derivatives.

            \item Taylor's formula for multivariable functions; 
            definition of and conditions for the extrema of multivariable functions.
        
        \end{itemize}
    \end{chapter}
\end{course}

\begin{course}{Mathematics II}{https://www.fmf.uni-lj.si/en/study-physics/programmes/1fiz/2020/7000777/courses/521/}{7}
    \label{mathematics_2}

    \begin{chapter}{Matrices}
        \begin{itemize}
        
            \item Fundamentals of matrices: definition of a matrix; square, diagonal, symmetric, and upper and lower triangular matrices;
            sub-matrices and block matrices.

            \item Computations with matrices: definition and properties of matrix addition and scalar multiplication; definition and properties of matrix multiplication;
            the transpose, conjugate, and Hermitian conjugate of a matrix.

            \item The inverse of matrix: definition of an inverse and a left and right inverse; conditions for matrix invertibility.

            \item Elementary row and column operations and elementary matrices; row-equivalent and column-equivalent matrices; the row-echelon and reduced row-echelon form of matrix.
            
            \item Gaussian elimination: algorithms for Gaussian and Gauss-Jordan elimination; the rank of a matrix; application of Gauss-Jordan elimination to computing matrix inverses.

            \item Systems of linear equations: Matrix formulation of a system of linear equations; homogeneous and associated homogeneous systems of equations; homogeneous and particular solutions; existence of a solution to a system of linear equations; application of Gauss-Jordan elimination to solving systems of linear equations.

            \item Permutations: definition and properties of permutations and cycles; the symmetric group of permutations; decomposition of permutations into compositions of disjoint cycles and transpositions; the sign of a permutation; even and odd permutations.

            \item The determinant of a matrix: definition of the determinant in terms of symmetric permutations; properties of the determinant; computation of the determinant for triangular and diagonal matrices; transformation of the determinant under elementary row and column operations; relationship of the determinant and invertibility; row and column expansion of determinants; Cramer's rule; minors and principal minors.
        
        \end{itemize}
    \end{chapter}

    \begin{chapter}{Finite-dimensional vector spaces}
        \begin{itemize}
        
            \item Definition of a vector space over a scalar field; definition and properties of the zero element and additive inverse; closed subsets of vector spaces.

            \item Frames: linear combinations of vectors; definition and properties of the span of a set of vectors; definition and properties of a frame for a vector space; 

            \item Bases: definition and theory of linearly independent and linearly dependent vectors; definition of the basis for a vector space; countable and ordered basis; conditions that a set of vectors form a basis; the standard basis.

            \item Linear transformations: definition and theory of linear transformations; compositions of linear transformations; monomorphic, epimorphic, and isomorphic transformations; endomorphisms and automorphisms of vector spaces; the kernel and image of linear transformation; transformation of bases; matrix representation of linear transformations.

            \item Dimension: definition of infinite-dimensional and finite-dimensional vector spaces; 
            isomorphism of an $ n $-dimensional vector space to $ \mathbb{C}^{n} $;
            relationship of dimension to frames and bases;
            dimension of vector subspaces.

            \item Vector subspaces: dimension of a vector subspace; affine subspaces; definition and properties of the intersection, sum, and direct sum of vector spaces; complementary vector spaces.

            \item Rank of a linear transformation: relationship of the rank, dimension, and image of a matrix; definition and properties of the rank and nullity of a linear transformation; the rank-nullity theorem.

        \end{itemize}
    \end{chapter}

    \begin{chapter}{Endomorphisms of vector spaces}
        \begin{itemize}
            \item Characteristic polynomial: definition and properties of matrix similarity; relationship between similarity, trace, and the determinant of a matrix; the determinant and trace of an endomorphism; the characteristic polynomial of an endomorphism.

            \item Eigenvalues and eigenvectors: endomorphism-invariance of a vector subspace; definition and properties of the eigenvalues of an endomorphism; algebraic and geometric multiplicity of an eigenvalue; definition and properties of the eigenvectors and eigenspaces of an endomorphism; linear independence of eigenvectors.

            \item Diagonalizability: definition of the coordinate matrix of an endomorphism in a given basis; definition and properties of diagonalizable endomorphisms; relationship between diagonalizability and multiplicity of eigenvalues.

            \item The Schur decomposition; Schur's theorem and its consequences; the Cayley-Hamilton theorem; functions of endomorphisms.
        
        \end{itemize}
    \end{chapter}

    \begin{chapter}{Inner product spaces}
        \begin{itemize}
        
            \item Fundamental concepts: the inner product of vectors; definition of an inner product space; norm of a vector and the distance between vectors; norms induced by inner products; orthogonality and orthogonal projections;

            \item Theorems arising from the inner product: the Pythagorean theorem, parallelogram rule, triangle inequality, and Cauchy-Schwartz inequality; the polarization equation.

            \item Orthonormal bases: normalization of vectors; definition and properties of an orthonormal basis for a finite-dimensional vector space; expansion and norm of a vector in an orthonormal basis; orthogonal projection onto a subspace; the Gram-Schmidt algorithm for orthogonalization of a basis; existence of an orthonormal basis; definition and properties of the orthogonal complement to a vector subspace.

            \item Dual spaces: definition and properties of linear functionals; definition and properties of the dual space; dual bases; the Riesz representation theorem.

        \end{itemize}
    \end{chapter}
        
    \begin{chapter}{Endomorphisms of inner product spaces}
        \begin{itemize}

            \item Definition and properties of adjoint operators; image and kernel of an adjoint operator; relationships between endomorphisms and their adjoints; definition and properties of linear isometries and isometric isomorphisms.

            \item Unitary operators: definition and properties of orthogonal and unitary operators; orthogonal (unitary) similarity; diagonalization of orthogonal (unitary) operators and matrices; examples of orthogonal transformations: rotations, reflections, and compositions of rotations and reflections.

            \item Normal matrices and endomorphisms; diagonalization and eigenspaces of normal endomorphisms.

            \item Definition and properties of self-adjoint endomorphisms; matrix representation of self-adjoint operators; eigenvalues, eigenvectors, and eigenspaces of self-adjoint endomorphisms and matrices

            \item Definition of a positive and negative-(semi)definite matrix and operator; properties of positive-definite endomorphisms; Sylvester's criterion for a positive-definite matrix.
        
        \end{itemize}
    \end{chapter}

    \begin{chapter}{Quadratic forms}
        \begin{itemize}
        
            \item Definition and properties of bilinear forms on vector spaces over the real numbers; symmetric bilinear forms; definition and properties of quadratic forms; matrix representation of symmetric bilinear forms; relationship between self-adjoint endomorphisms and symmetric bilinear forms.

            \item Diagonalization of a bilinear form and its associated quadratic form; the coefficient matrix of a symmetric bilinear form in a given basis; Sylvester's law of inertia for real quadratic forms; index of positivity, index of negativity, signature, and rank of a symmetric bilinear form.

            \item Definition and properties of curves, surfaces, and second-order hyperplanes and their representation with quadratic forms; degenerate and non-degenerate hyperplanes; classification of quadratic surfaces.
        
        \end{itemize}
    \end{chapter}
 
    \begin{chapter}{Vector functions of several variables}
        \begin{itemize}
        
            \item Metric spaces: definition and properties of a metric and a metric space; metric subspaces; definition and properties of open and closed sets and neighborhoods and of the interior, closure and boundary of metric spaces; properties of continuous mappings between metric spaces; definition and properties of compact Euclidean metric spaces; path-connected metric spaces and Banach's fixed-point theorem.

            \item Calculus of vector-valued functions:limits and continuity of vector valued functions: 
            partial differentiability and partial derivatives; the gradient operator and directional derivatives;
            the total derivative, Jacobian matrix, and Jacobian determinant of a vector-valued function.

            \item The inverse and implicit function theorems.

            \item Constrained extrema and the method of Lagrange multipliers.

        \end{itemize}
    \end{chapter}
\end{course}

\begin{course}{Classical Physics}{https://www.fmf.uni-lj.si/en/study-physics/programmes/1fiz/2020/7000777/courses/1154/}{20}
    \label{classical_physics}

\begin{chapter}{Mechanics}

    \begin{itemize}
    
        \item Kinematics of a point mass: position, velocity, and acceleration; kinematics of free fall and projectile motion; rotational kinematics; uniform and general circular motion; Galilean relativity.

        \item Dynamics of a point mass: force, definition of inertial frames of reference, Newton's laws of motion; common forces from daily life: weight, friction, the normal force, spring force, tension; generalization of Newton's laws in non-inertial frames of reference, pseudo-forces.
     
        \item Energy: work and kinetic energy; the work-energy theorem; conservation of energy; potential energy and conservative forces; power.

        \item Linear momentum and conservation of linear momentum.

        \item Rotational dynamics: torque, moment of inertia, the parallel axis theorem; equilibrium and dynamics of rigid bodies; rotation about a fixed axis and rolling without slipping; combined kinetic energy of rotation and translation.

        \item Angular momentum and conservation of angular momentum

        \item Deformations and elasticity

        \item Oscillation: simple harmonic motion, Hooke's law, and the motion of pendulums for small oscillations about equilibrium; damped and forced oscillations, resonance.

        \item Gravitation: Newton's law of gravitation; general form of gravitational potential energy; conservative forces; conservation of energy and angular momentum under central force motion; Kepler's laws of planetary motion.

        \item Introductory fluid mechanics: static fluids, flow of non-viscous and viscous fluids, Bernoulli's equation, motion in fluids. 
    \end{itemize}

\end{chapter}

\begin{chapter}{Mechanical waves}
    \begin{itemize}
        \item The wave equation and propagation of mechanical disturbances in elastic media; waves in one, two and three dimensions.

        \item Reflection, refraction, diffraction, and interference of waves; superposition of waves and standing waves.

        \item Energy and power in wave motion.

        \item The Doppler effect.

        \item Wave dispersion.
        
    \end{itemize}
\end{chapter}

\begin{chapter}{Introduction to classical thermodynamics}
    \begin{itemize}
    
        \item Thermodynamic systems and variables, thermodynamic equilibrium; temperature the definition of temperature scales
        \item Heat, specific and latent heat, and heat transfer. 
        \item Equations of state, phase transitions, and phase diagrams.
        \item Internal energy and enthalpy
        \item The first and second laws of thermodynamics; entropy, reversible and irreversible processes, cycles; heat engines and refrigerators.
        \item Thermodynamics of ideal gases; the kinetic theory of gases; the Maxwell-Boltzmann distribution and statistical description of kinetic theory. 
    
    \end{itemize}
\end{chapter}

\begin{chapter}{Electrostatics}
    \begin{itemize}

        \item Electric charge; the electrostatic force, electric field and the electric potential; Gauss’s law.
        
        \item Conductors and dielectrics in an external electric field. 

        \item The capacitor; electric energy and electric energy density. 

        \item The microscopic theory electric current and conduction in metals.

        \item Electric circuits: voltage, current and electric power; voltage and current sources; Ohm’s law and Kirchhoff’s laws; the voltmeter and ammeter; alternating current circuits.

    \end{itemize}
\end{chapter}

\begin{chapter}{Magnetostatics and magnetic induction}
    \begin{itemize}
    
        \item Mechanisms for the magnetic field: charged particles in motion, and electric current; permanent magnets.

        \item The magnetic force and the Lorentz force on charged particle in a magnetic field; magnetic flux density; magnetic torque on a magnetic dipole.

        \item The Ampere and Biot-Savart laws; Gauss's law for the magnetic field.

        \item The inductor, magnetic induction, Lenz's law, self-induction, mutual induction; magnetic field energy.

        \item Circuits with inductors: electric oscillators, RL and RLC circuits; simple model of the radio receiver.

        \item Magnetism of matter: diamagnetic, paramagnetic, and ferromagnetic materials. 
    \end{itemize}
\end{chapter}

\begin{chapter}{Electromagnetic wave propagation}
    \begin{itemize}

        \item Maxwell’s equations in free space.

        \item Electromagnetic energy and power; the Poynting vector.

        \item The dipole antenna and Hertz radiation.

        \item Confined wave propagation: the coaxial cable, waveguides, and impedance.

        \item Topics from wave optics: reflection and refraction; interference and diffraction; the index of refraction; dispersion.

        \item Polarization of electromagnetic waves and optically anisotropic matter.

    \end{itemize}
\end{chapter}

\begin{chapter}{Geometrical optics}
    Postulates of geometrical optics; magnification and focus; common optical instruments: mirrors, lenses, microscopes, telescopes, the human eye.
\end{chapter}

\end{course}

\begin{course}{Proseminar A}{https://www.fmf.uni-lj.si/en/study-physics/programmes/1fiz/2020/7000777/courses/1170/}{3}
    \label{proseminar_a}

    \begin{chapter}{Material}
        \begin{itemize}
            
            \item Course summary: the use of vector algebra as well as differential and integral calculus for solving advanced problems from Classical Physics.

            \item The Cartesian, cylindrical, and spherical coordinate systems; example: analysis of circular and helical motion in a cylindrical basis and coordinate system;

            \item Vector calculus in cylindrical and spherical coordinates; differential area and volume elements; transformation of coordinates in integration; line and surface integrals; example: Ampere's law.

            \item Ordinary first and second-order differential equations: separation of variables and exact equations; homogeneous and particular solutions.

            \item Oscillation: analysis of oscillation with complex exponentials; simple harmonic motion; damped oscillation; coupled differential equations and the example of the coupled pendulum.

            \item Introduction the Lagrangian formalism: the Lagrangian function; the Euler-Lagrange equations; examples: Lagrangian analysis of a coupled pendulum and of the vibration of the $ \mathrm{CO}_{2} $ molecule.

        \end{itemize}
    \end{chapter}

\end{course}

\begin{course}{Chemistry I}{https://www.fmf.uni-lj.si/en/study-physics/programmes/1fiz/2020/7000777/courses/1152/}{3}
    \label{chemistry_1}

    \begin{chapter}{Material}
        \begin{itemize}
            
            \item Introduction to the microscopic composition and structure of matter.

            \item Nomenclature of simple compounds.

            \item Stoichiometric laws; Quantitative information from balanced equations; limiting reactant and theoretical yield of reactions.

            \item Enthalpies of reaction; Hess's law; calculation of enthalpy changes.

            \item Atomic structure; quantum numbers and orbitals;electron configurations and the structure of the periodic table.

            \item Chemical bonds and the geometry of simple molecules; interactions between molecules.

            \item Types of crystalline solids and their properties.

            \item Liquids and solutions: colligative properties, electrolytes, dissociation, adsorption.

            \item Reaction rates and reaction kinetics; influence of concentration and temperature; the Arrhenius equation; catalysis.

            \item Chemical equilibrium; Le Chatelier's principle; ionic reactions and solubility equilibria.

            \item Definitions of acids and bases; pH; strong and weak acids and bases; buffers and indicators.

            \item Oxidation and reduction, standard electrode potentials, and electrochemistry.

            \item Coordination compounds, coordination numbers, coordination polyhedra, isomerism, and bonding in coordination species.

            \item Introduction to organic chemistry: types of organic compounds and typical reactions of organic compounds.
            
        \end{itemize}
    \end{chapter}

\end{course}

\begin{course}{Computational Tools in Physics}{https://www.fmf.uni-lj.si/en/study-physics/programmes/1fiz/2020/7000777/courses/1172/}{3}
    \label{computational_tools_in_physics}

    \begin{chapter}{Material}
        \begin{itemize}
            
            \item Plotting: 2D plots and trajectories; histograms; log and log-log plots; plotting a family of curves; 3D plots; false-color presentation.

            \item Software: \LaTeX; the scientific Python stack (\texttt{Numpy}, \texttt{Matplotlib}, \texttt{SciPy}); Jupyter notebooks; Bash shell scripting.

            \item Linear and non-linear regression; least squares error minimization; linear and non-linear curve fitting with the SciPy library.
        
            \item Numerical differentiation; symmetric and asymmetric finite differences; Taylor expansions and error bounds.

            \item Numerical integration: the trapezoidal rule and Simpson's rule.

            \item Numerical methods for ordinary differential equations: explicit and implicit Euler method, Crank-Nicholson method, Runge-Kutta methods; solving higher-order and vector-valued ODEs.

            \item Distributions: discrete and continuous distributions; binning and histograms; computing quantiles and distribution parameters, e.g. mean, variance, median, and higher-order moments.

            \item Data processing: tabulation of data; sorting; noise reduction; one-pass operations: sums, scalar products, and averages.

            \item Correlation: the correlation coefficient; cross-correlation and auto-correlation; applications of auto-correlation to analyzing noisy periodic signals.

        \end{itemize}
    \end{chapter}

\end{course}

\begin{course}{Physics Laboratory I}{https://www.fmf.uni-lj.si/en/study-physics/programmes/1fiz/2020/7000777/courses/1142/}{4}
    \label{physics_laboratory_1}

    \begin{chapter}{Experiments}
        \begin{itemize}
        
            \item Magnitude of the gravitational acceleration $ g $

            \item Bernoulli's equation

            \item The moment of inertia of a rotating rigid body

            \item Hooke's law and the elastic modulus; regimes of linearity, elasticity, and plasticity in a copper wire

            \item Viscosity

            \item The specific heat of solids

            \item The dynamics of a coupled pendulum

            \item Ultrasound and the speed of sound.
        
        \end{itemize}
    \end{chapter}

\end{course}

\begin{course}{Physics Laboratory II}{https://www.fmf.uni-lj.si/en/study-physics/programmes/1fiz/2020/7000777/courses/1143/}{4}
    \label{physics_laboratory_2}

    \begin{chapter}{Experiments}
        \begin{itemize}
        
            \item Wheatstone bridge

            \item Driven oscillations of a resonant circuit

            \item Magnetic force on a current-carrying conductor

            \item Torque on a coil in a magnetic field

            \item Force between capacitor plates

            \item Transient phenomena in electrical circuits

            \item Measurement of diffraction grating spectra

            \item The photoelectric effect

            \item Absorption of gamma rays
        
        \end{itemize}
    \end{chapter}
\end{course}

\begin{course}{Computing Laboratory}{https://www.fmf.uni-lj.si/en/study-physics/programmes/1fiz/2018/7000777/courses/525/}{3}
    \label{computing_laboratory}

    \begin{chapter}{Material}
        \begin{itemize}
        
            \item Computing environment at the faculty (electronic mail, online courses and other services, software available to students, accessing computer labs)

            \item Basics of operating systems: file systems, backups, networking, computer security.

            \item Introduction to the C programming language: variables, conditional statements, loops, pointers, arrays, standard input/output, reading and writing files.
        
        \end{itemize}
    \end{chapter}
\end{course}

\begin{course}{How Things Work}{https://www.fmf.uni-lj.si/en/study-physics/programmes/1fiz/2020/7000777/courses/1151/}{3}
    \label{how_things_work}

    \begin{chapter}{Material}
        \begin{itemize}
        
            \item Course summary: learning and application of basic physics concepts and laws in the context of materials and technology from everyday life.

            \item Material includes: internal combustion engine, LCD monitor, fluorescent lamp, halogen bulb, LED, magnetic and piezoelectric speaker, MRI.
        
        \end{itemize}
    \end{chapter}
\end{course}

\newpage
\section{Year 2}
\begin{course}{Mathematics III}{https://www.fmf.uni-lj.si/en/study-physics/programmes/1fiz/2020/7000777/courses/522/}{8}
    \label{mathematics_3}

    \begin{chapter}{Parameter-dependent integrals}
        \begin{itemize}
        
            \item Definition of parameter-dependent integrals

            \item Continuity and differentiability of parameter-dependent integrals; Fubini's theorem.

            \item Improper parameter-dependent integrals: convergence and uniform convergence, the Weierstrass test for uniform convergence; continuity, differentiability, and integrability.

            \item The gamma and beta function: definitions, properties, relationship, trigonometric identities; Stirling's formula and approximation.
        
        \end{itemize}
        
    \end{chapter}

    \begin{chapter}{Multi-dimensional Riemann integration}
        \begin{itemize}
        
            \item Partitions, upper and lower sums, Riemann sums; Riemann integrability; theorem: continuity implies integrability.

            \item Evaluation of iterated integrals in Cartesian, cylindrical, and spherical coordinates; change of variables between coordinate systems.

            \item Volume: definition of volume in $ \mathbb{R} $ and $ \mathbb{R}^{N} $; condition for existence of $ N $-dimensional volume; relationship between subsets of $ \mathbb{R}^{N} $ with zero measure and zero $ N $-volume.

            \item Applications of double and triple integrals: computing volume, mass, center of mass, moment of inertia.
        
        \end{itemize}
    \end{chapter}

    \begin{chapter}{Curves and surfaces in Euclidean space}
        \begin{itemize}
        
            \item Curves in $ \mathbb{R}^{3} $; parameterizations; line integrals and arc length.

            \item The tangent, normal, and binormal vectors; normal and osculating planes; the Frenet-Serret frame.

            \item Curvature and radius of curvature; torsion; the Frenet-Serret formulae.

            \item Surfaces in $ \mathbb{R}^{3} $; implicitly, explicitly, and parametrically-defined surfaces; normal vector and tangent plane.

            \item Surface area and surface integrals; surfaces of revolution.
        
        \end{itemize}

    \end{chapter}

    \begin{chapter}{Vector calculus}
        \begin{itemize}
        
            \item Line integrals of scalar and vector fields.

            \item The gradient of a vector field; conservative vector fields; the gradient theorem for line integrals.

            \item Surface integrals of scalar and vector fields.

            \item Vector operators: the curl, divergence, and Laplacian operators.

            \item Important theorems in vector calculus: The divergence (Gauss-Ostrogradsky) theorem, Green's theorem, Stoke's theorem, Green's identities.

            \item Vector operators in spherical and cylindrical coordinate systems.
        
        \end{itemize}
        \end{chapter}

        \begin{chapter}{Ordinary differential equations}
            
            \begin{itemize}
            
                \item Ordinary, first-order LDEs: homogeneous and exact LDEs; separation of variables; the first integral of an LDE; integrating factors; parametric differential equations; the Clairaut differential equation.

                \item Review of metric spaces; Lipshitz continuity; Banach's fixed-point theorem; Picard's existence theorem.

                \item Systems of LDEs: homogeneous and non-homogeneous systems of LDEs with constant and variable coefficients; dimension of the solution set; the fundamental matrix of a system of LDEs; particular and general solutions for non-homogeneous systems of LDEs; the Wronskian determinant.

                \item Higher-order LDEs with constant coefficients: operator notation; trial solution and characteristic polynomial.

                \item Higher-order LDEs with variable coefficients: matrix representation; dimension of the solution set; solution using the Wronskian matrix.

            \end{itemize}

        \end{chapter}

        \begin{chapter}{Calculus of variations}
            \begin{itemize}
                
                \item Definition: functionals and the Lagrangian.

                \item The Euler-Lagrange equations.

                \item Critical point of a functional and necessary condition for extrema.

                \item Examples: the brachistochrone problem and isoperimetric problems; special cases with a Lagrangian independent of $ x $ or $ y $; calculus of variation problems with variable endpoints.

                
            \end{itemize}
        \end{chapter}

        \begin{chapter}{Introduction to Hilbert spaces}
            \begin{itemize}
            
                \item Definitions: inner product, inner product space, norm, metric, and completeness.
                Definition of a Hilbert space.

                \item Orthogonality, the orthogonal complement, closure of the orthogonal complement.

                \item Orthogonal and oblique projection; the projection theorem.

                \item The Cauchy-Schwartz inequality and the Bessel inequality.

            \end{itemize}

        \end{chapter}

        \begin{chapter}{Introduction to Fourier series}
            \begin{itemize}
            
                \item The $ \mathcal{L}^{2}[-\pi, \pi] $ and $ \mathcal{L}^{2}[-a, a] $ Hilbert spaces; complete orthonormal bases for $ \mathcal{L}^{2}[-\pi, \pi] $.

                \item Definition and properties of Fourier series; properties of Fourier coefficients.

                \item Convergence of Fourier series under the $ \mathcal{L}^{2} $ norm.

                \item Pointwise convergence of Fourier series for continuous and piecewise-continuous functions.
            \end{itemize}
        \end{chapter}
\end{course}

\begin{course}{Mathematics IV}{https://www.fmf.uni-lj.si/en/study-physics/programmes/1fiz/2020/7000777/courses/523/}{6}
    \label{mathematics_4}

    \begin{chapter}{Complex analysis}
        \begin{itemize}
        
            \item The extended complex plane

            \item Differentiability in complex analysis: complex differentiation; differentiation of complex power series; convergence of complex power series.

            \item The Cauchy-Riemann equations, relationship between harmonic and holomorphic functions.

            \item Integration of complex functions: contour integration; parameterization of complex contours, connected combinations of contours, and closed contours; the fundamental theorem of calculus for contour integration.

            \item The winding number and Cauchy's theorem: definition and properties of the winding number; Cauchy's theorem, Cauchy's integral and differentiation formula, Cauchy's inequality for derivatives; Liouville's theorem and the fundamental theorem of algebra.

            \item Laurent series: definition and properties of Laurent series, the principle and regular parts of Laurent series, relationship between Taylor and Laurent series.

            \item Singularities of complex functions: definition of isolated, removable, and essential singularities; poles of complex functions; the Casorati-Weierstrass theorem; definition of meromorphic functions.

            \item Definition, integration, and differentiation of the complex power and logarithm functions; branches of the complex logarithm.

            \item Residue: definition of the residue of a complex function; the residue theorem; computing residue at zeros and poles; examples of integration using the residue theorem.

            \item The open mapping theorem and maximum modulus principle: definition of an open map; poles and zeros of meromorphic functions definition on a disk; the open mapping theorem; the bijectivity of holomorphic maps; the maximum modulus principle; the Schwarz lemma; biholomorphisms and automorphisms of the unit disk; the Moibious transform.
        
        \end{itemize}
    \end{chapter}

    \begin{chapter}{Harmonic functions}
        \begin{itemize}
        
            \item Harmonic function in the plane: the Poisson kernel and Poisson formula on the unit disk and on arbitrary disks; properties of the Poisson kernel; the mean value property and maximum principle in the plane; the Dirichlet problem on the unit disk.

            \item Harmonic function in space: fundamental solution of the Laplace equation in space; review of Gauss's and directional derivatives; Green's identities in space; the mean value property and maximum and minimum principle in space; the Dirichlet problem in space; Green's functions in space; formulation of the Poisson kernel and formula in terms of Green's functions.
        
        \end{itemize}
    \end{chapter}

    \begin{chapter}{Fourier analysis}
        \begin{itemize}
        
            \item Convolution: definition and properties of convolution; differentiating convolution;
            convolution of functions in $ \mathcal{L}^{1}(\mathbb{R}) $; definition of compact support; the function $ g_{\delta}(x) = \frac{1}{\delta} g(x/\delta) $ where $ g \in \mathcal{L}^{1}(\mathbb{R}) $ and $ \delta > 0 $; convergence of convolutions involving $ g_{\delta} $;
            approximation by sequences of smooth functions and the Weierstrass approximation theorem; definition and properties of the Schwartz space $ \mathcal{S}(\mathbb{R}) $ of rapidly-decreasing functions.

            \item The Fourier transform in $ \mathbb{R} $: definition of the Fourier transform; properties of the Fourier transform; the inverse Fourier transform; the Riemann-Lebesgue Lemma; the Fourier transform and Lipshitz continuity; Plancherel’s theorem.

            \item The Fourier transform in $ \mathbb{R}^{N} $: definition of the spaces $ \mathcal{L}^{1}(\mathbb{R^{N}}) $ and $ \mathcal{L}^{2}(\mathbb{R^{N}}) $; convolution in $ \mathbb{R}^{N} $; the Schwartz space of rapidly-decreasing functions in $ \mathbb{R}^{N} $; definition and properties of the Fourier transform in $ \mathbb{R}^{N} $; the inverse Fourier transform in $ \mathbb{R}^{N} $; the Plancherel theorem and Riemann-Lebesgue lemma for functions in $ \mathbb{R}^{N} $.

        \end{itemize}
    \end{chapter}

    \begin{chapter}{Partial differential equations}

        \begin{itemize}
            
            \item Definition of the heat equation; application of the Fourier transform to the heat equation.

            \item Vibration of strings and the wave equation: solving the one-dimensional wave equation with the Fourier method; d'Alembert's formula for solving the wave equation.

            \item Vibration of circular membranes
            
        \end{itemize}
    \end{chapter}

    \begin{chapter}{Sturm-Liouville theory and second-order homogeneous LDEs}
        \begin{itemize}
        
            \item Zeros of solutions to second-order homogeneous LDEs: shared zeros and linear (in)dependence; normal form of a second-order LDE; Sturm's comparison criterion; example: zeroes of the Bessel equation.

            \item Linear differential operators: the space of continuous and twice-differentiable functions on $ [a, b] $; definition of linear differential operators, adjoint of a differential operation, formally self-adjoint differential operators.

            \item The Sturm-Liouville problem: definition of the Sturm-Liouville problem; the SL problem with a weight; inner product, norm, and orthogonality of functions with a weight; eigenvalues and eigenfunctions of self-adjoint differential operators; the regular Sturm-Liouville problem; the Sturm-Liouville theorem.
        
        \end{itemize}
        
    \end{chapter}

    \begin{chapter}{Power series solutions of second-order DEs}

        \begin{itemize}
        
            \item The Frobenious method for power series solutions of second-order differential equations

            \item Proper singular points of differential equations; behavior of series solutions near proper singular points.

            \item General orthogonal polynomials: zeros and oscillation of orthogonal polynomials; orthogonal polynomials as a basis for $ \mathcal{L}^{2}(-1, 1) $.

            \item The Legendre equation: definition of the Legendre equation; the Legendre polynomials and solutions to the Legendre equation; the associated Legendre polynomials; generating function and the Rodrigues formula; orthogonality and norms of the Legendre polynomials.

            \item The Bessel equation: definition of the Bessel equation; the Bessel functions; the generating formula for the Bessel functions $ J_{n} $ for $ n \in \mathbb{N} $; relationship of the generating function and the integral formula for the Bessel functions.

            \item The Hermite and Laguerre polynomials: generating functions; norms and orthogonality; recurrence relations.
        
        \end{itemize}
        
    \end{chapter}

\end{course}

\begin{course}{Modern Physics I}{https://www.fmf.uni-lj.si/en/study-physics/programmes/1fiz/2020/7000777/courses/1161/}{8}
    \label{modern_physics_1}

    \begin{chapter}{Special theory of relativity}

        \begin{itemize}
        
            \item Review of the Galilean transformations and Galilean relativity; inertial and non-inertial frames of reference; the Michelson-Morley experiment.

            \item The speed of light and the fundamental postulates of relativity

            \item Space-time, time dilation and length contraction; the Lorentz transformation for position, time, and velocity.

            \item Relativistic energy and momentum: four-vectors; definitions of relativistic momentum and energy; the scalar product of four-vectors; conservation of energy and momentum; rest energy and relativistic kinetic energy.

            \item The space-time and four-vector formalism: generalization of Euclidean space to four-dimensional space-time; intervals in space-time; invariance of space-time length; contravariant and covariant four-vectors; the Lorentz transformation matrix; four vectors for velocity and momentum; Lorentz transformation of the electric and magnetic field.

            \item Relativistic dynamics: four-vector for relativistic force; the one-dimensional work-energy theorem in special relativity; relativistic particle in an electromagnetic field.


        \end{itemize}
        
    \end{chapter}

    \begin{chapter}{Quantum physics}

        \begin{itemize}
        
            \item Motivation for quantum physics: the stability of atoms, black-body radiation, the photoelectric effect.

            \item Particles as waves: the wave function and probability amplitude; the uncertainty principle.

            \item Physical quantities and operators on wave functions.

            \item The Schroedinger equation; eigenvalues and eigenstates of the Hamiltonian operator; stationary states;  expansion of a wavefunction in energy eigenstates.

            \item Examples: Particle in an infinite and finite potential well and the harmonic oscillator.

            \item Probability current: definition of probability current density, particles in piecewise continuous potentials; quantum tunneling through potential barriers.

            \item The Schroedinger equation in 3 dimensions and the particle in a three-dimensional box. 

            \item Angular momentum: commutation rules; eigenvalues and eigenfunctions of the angular momentum operator; example of the quantum rotator.

            \item The hydrogen atom: energy eigenvalues and eigenstates.

            \item Spin; orbital magnetic moment and the Stern-Gerlach experiment; addition of angular momentum and spin-orbit coupling; the Zeeman effect; two-state systems.

            \item Radiative transitions: selection rules; width of spectral lines; Fermi's golden rule; the visible and X-ray spectra.

            \item Multiple-electron atoms: single particle states; the Pauli exclusion principle; the periodic table of elements; 

            \item Chemical bonds: the ionic, covalent, and Van Der Walls bond; molecular vibrations and rotations; vibrational and rotational spectra.

        \end{itemize}
    \end{chapter}

\end{course}

\begin{course}{Modern Physics II}{https://www.fmf.uni-lj.si/en/study-physics/programmes/1fiz/2020/7000777/courses/1162/}{5}
    \label{modern_physics_2}

    \begin{chapter}{Introduction to solid state physics}
        \begin{itemize}
        
            \item Energy bands in solids: review of the energy levels of the one-electron atom; electron levels in the presence of multiple nuclei; energy bands in a periodic potential; the Kronig-Penney model

            \item Bloch's theorem in one dimension.

            \item Electron conduction:  review of current density, conductivity, and mobility; the Drude model equation of motion, the Drude model in a constant and alternating electric field; predictions for conductivity.

            \item Review of fermion statistics: the Fermi function; Fermi energy and chemical potential; electron density of states.

            \item Intrinsic semiconductors: approximate Fermi functions for electrons and holes; effective mass; charge carrier density of state; the Fermi level and energy gap.
            \item Doped semiconductors: donor energy levels; approximate Fermi functions for donor and intrinsic electrons and holes; charge carrier density of state; the Fermi level in doped semiconductors.

            \item The p-n junction: the depletion region and formation of the p-n junction; electric potential and electric field across the p-n junction; the chemical potential across the p-n junction; width of the depletion region.

            \item Applications of the p-n junction: forward and reverse biasing; the p-n junction diode; the photodiode; the bipolar transistor; current-voltage characteristics for the p-n junction diode and bipolar transistors.

        \end{itemize}
        
    \end{chapter}

    \begin{chapter}{Introduction to nuclear physics}
        \begin{itemize}
        
            \item Nuclei as the building blocks of matter

            \item The Rutherford experiment: review of the central force problem; the Rutherford cross section; implications of the Rutherford experiment for the existence, size, and mass of nuclei.

            \item Nuclear mass and composition: neutron, proton, and mass numbers; model of the nuclear radius; binding energy; the semi-emperical mass formula and its contributing factors.

            \item The shell model of the nucleus: the Schroedinger equation and the shell model; magic numbers and nuclear stability; the periodic table; spin-orbit coupling and nuclear spin in the scope of the shell model.

            \item Nuclear reactions and decay: alpha, beta, and gamma decay; nuclear fission and the example of the uranium nucleus; induced and spontaneous fission; nuclear fusion; the Lawson criterion.
        
        \end{itemize}
    \end{chapter}

    \begin{chapter}{Introduction to particle physics}
        \begin{itemize}
        
            \item Basics of charged particle acceleration using electromagnetic fields: linear, cyclotron, and synchrotron accelerators.

            \item Basics of particle detection: cross section; determination of momentum; measurement of mass and the use of mass for particle identification.

            \item Introduction to the Standard Model: categorization of particles, into e.g. leptons, quarks, bosons, hadrons, baryons and mesons; properties of particles, e.g. spin, mass, electric charge, and interactions; baryon octets and decuplets; the quark model of hadrons and color charge.

            \item Quantum numbers and conservation laws: charge; lepton and baryon numbers; isospin; strangeness.
            

            \item Coupling constants: the Compton wavelength; the Yukawa potential and transition matrix elements; probability and range of transitions and interactions.

            \item Compton wavelength; coupling constants of interactions. Lepton number, baryon number, isospin, strangeness. 

            \item The weak interaction; the parity, charge conjugation, and helicity operators; violation of charge symmetry under the weak interaction; conservation of charge-parity symmetry; the cobalt-60 experiment; the Cabibo angle and the CKM matrix.

            \item Feynman diagrams: rules and notation for drawing Feynman diagrams; examples of Feynman diagrams for common interactions; estimation of branching ratios with Feynman diagrams.

        \end{itemize}
    \end{chapter}

\end{course}

\begin{course}{Classical Mechanics}{https://www.fmf.uni-lj.si/en/study-physics/programmes/1fiz/2020/7000777/courses/1155/}{5}
    \label{classical_mechanics}

    \begin{chapter}{Review of Newtonian mechanics}
        \begin{itemize}
        
            \item Newtonian mechanics of a single particle: momentum and force; angular momentum and torque; work, energy, and conservative forces.

            \item Newtonian mechanics of systems of particles: center of mass; momentum and force on a system of particles; angular momentum and torque on a system of particles; work and energy; the virial theorem.

            \item Non-inertial systems and fictitious forces: inertial and non-inertial frames; transformation of coordinates and basis vectors; fictitious forces Newton's second law in non-inertial systems; rotating coordinate systems: the centrifugal, Coriolis, and Euler forces.
        
        \end{itemize}
    \end{chapter}

    \begin{chapter}{Lagrangian mechanics}
        \begin{itemize}
        
            \item Constraints and generalized coordinates: coordinates and degrees of freedom; holonomic constraints and generalized coordinates; Lagrange multipliers; the Lagrange equations in generalized coordinates.

            \item The principle of least action: action and the Lagrangian; the principle of least action; derivation of the Lagrange equations from the principle of least action; the Lagrange equations in changing coordinate systems.

            \item The d'Alembert principle: virtual displacements and virtual work; d'Alembert's principle; derivation of the Lagrange equations from d'Alembert's principle.

            \item Symmetries and conserved quantities: constants of motion/conserved quantities; the Hamiltonian and conservation of energy, conservation of generalized momenta; continuous symmetries of the Lagrangian; statement and proof of Noether's theorem; examples with Noether's theorem: momentum conservation and the homogeneity of space, angular momentum conservation and the isotropy of space, energy conservation and the homogeneity of time.

            % \item Small oscillations and stability:
        
        \end{itemize}
    \end{chapter}

    \begin{chapter}{Dynamics of central force motion}
        \begin{itemize}
        
            \item The one-body central force problem: conservation of angular momentum and energy; reduction to planar motion; deriving the orbit equation with both Newtonian and Lagrangian dynamics.

            \item The two-body central force problem: conservation of energy and angular momentum; reduced mass and effective potential; reduction to an equivalent one-body problem; existence and stability of circular orbits.

            \item The Kepler problem: inverse square potentials; solving the orbit equation and analysis of the conic section solutions; energy of orbit; Kepler's laws of planetary motion.

            \item The Laplace-Runge-Lenz vector: derivation, conservation of the LRL vector in central force problems; alternate derivation of the orbit equation.
        
        \end{itemize}
    \end{chapter}

    \begin{chapter}{Rigid body dynamics}
        \begin{itemize}
        
            \item Description of the rigid body: definition; degrees of freedom; the inertia tensor.

            \item Describing rigid body motion: the inertial (lab) and rotating reference frames; Euler's rotation theorem and the parallel axis theorem; the Euler angles and transformation matrices; tensor formulation of angular velocity; the Euler equations for rigid body motion.

            \item The free top: the symmetric free top: Euler equations, inertia tensor, and motion; the asymmetric free top: Euler equations and inertia tensor; stability and perturbations.

            \item The heavy symmetric top: the Lagrangian and conserved quantities; reduced energy; uniform precession without nutation of the fast top; motion of the sleeping top.
        \end{itemize}
    \end{chapter}

    \begin{chapter}{Small oscillations}
        \begin{itemize}
        
            \item General concepts: Taylor expansion of potential energy to second order about an equilibrium position; approximation of motion using uncoupled harmonic oscillators.

            \item System with $ N $ degrees of freedom: kinetic and potential energy matrices; reduction to a simplified eigenvalue problem; eigenfrequencies and normal modes; motion and stability of each normal mode.

            \item The generalized eigenvalues problem; normal coordinates and transformation to normal coordinates.
        
        \end{itemize}
    \end{chapter}

    \begin{chapter}{Hamiltonian mechanics}
        \begin{itemize}
        
            \item The Hamiltonian formalism: the Hamiltonian and the Legendre transform relating the Lagrangian and Hamiltonian; conjugate momenta and Hamilton's equations of motion.

            \item Hamiltonian mechanics and conservation laws: conservation of energy for time-independent Hamiltonians; conservation of conjugate momenta for cyclic coordinates.

            \item Hamiltonian formulation of the least action principle: definition of action in terms of the Hamiltonian; derivation of Hamilton's equations from the principle of least action.

            \item Hamiltonian and Lagrangian functions of a charged particle in an electromagnetic field.

            \item Poisson brackets: definition and properties of Poisson brackets, commutation and the Jacobi identity; Poisson brackets and angular momentum and overview of application to quantum mechanics.

        \end{itemize}
    \end{chapter}

    \begin{chapter}{Introduction to continuum mechanics}
        \begin{itemize}
        
            \item Lagrangian formalism for one-dimensional continuous bodies: the Lagrangian and Lagrange energy density; action and the Lagrange equations; generalization of momenta and energy to continuous media.

            \item Lagrangian formalism for one-dimensional continuous bodies: generalization of the Hamiltonian to continuous media; action and the Hamilton equations.

            \item Example: longitudinal vibrations of an elastic rod.

        \end{itemize}
    \end{chapter}
\end{course}

\begin{course}{Statistical Thermodynamics}{https://www.fmf.uni-lj.si/en/study-physics/programmes/1fiz/2020/7000777/courses/1174/}{5}
    \label{statistical_thermodynamics}

    \begin{chapter}{Part 1: Classical thermodynamics}
        \begin{itemize}
            \item Equations of state: thermodynamic equilibrium and the zeroth law of thermodynamics; temperature scales: the gas thermometer and Carnot cycle; state variables and equations of state; the ideal gas equation and the Van Der Waals equation.

            \item The first law of thermodynamics: statement of the first law; enthalpy; heat capacity; examples: Joule expansion and the Joule-Thomson-Kelvin effect.

            \item The second law of thermodynamics: quasi-static, reversible, and irreversible processes; statement of the second law; the Carnot cycle and the Carnot cycle temperature scale; entropy and implications of the definition of entropy; entropy of a ideal gas.

            \item Thermodynamic potentials: internal energy, enthalpy, Helmholtz free energy, and Gibbs free energy; Maxwell's relations; applications of Maxwell's relations to $ (p, V, T) $ systems: coefficients of compressibility and thermal expansion, difference of heat capacities, and the Joule-Kelvin coefficient.

            \item Phase changes: definition of phase changes; the liquid-gas phase change; Gibbs free energy, chemical potential, and the conditions for phase equilibrium; the Maxwell construction; continuous and non-continuous phase changes; the Clausius-Clapeyron equation.

            \item Introduction to transport phenomena: diffusion of matter, the continuity and diffusion equations for diffusion of matter; diffusion of energy, the continuity and diffusion equations for heat transfer; cross transport phenomena: the thermoelectric effect and the Peltier effect.
            
        \end{itemize}
    \end{chapter}

    \begin{chapter}{Part 2: Statistical physics}
        \begin{itemize}
        
            \item Concepts in statistical physics: phase space; microstates, probability density, and equilibrium; entropy at fixed energy and the microcanonical ensemble; entropy and the statistical formulation of the second law; temperature and the statistical formulation of the zeroth law.

            \item The canonical ensemble: definition of the canonical ensemble; the canonical partition function; energy fluctuations in the canonical ensemble; Helmholtz free energy; the equipartition principle.

            \item Entropy: the Gibbs entropy formula in the microcanonical and canonical ensembles; the Boltzmann entropy formula; separation of energy levels in macroscopic systems; the two-state system.

            \item Equations of state and the virial expansion: equation of state and partition function for an ideal gas; the virial expansion and the virial equation of state; the second virial coefficient and derivation of the Van Der Waals equation from the virial expansion.

            \item Quantum statistical physics: the Pauli exclusion principles; wave functions for multi-particle systems; the Bohr-Sommerfeld quantization rule; density of states; bosons and fermions.

            \item Fermi-Dirac statistics: the Fermi function; the Fermi energy and Fermi sphere; internal energy, heat capacity, and chemical potential of a free electron gas; applications to electrons in metals and solid state physics.

            \item Bose-Einstein statistics: the Bose-Einstein distribution; statistics of the free boson gas; applications to phonons in metals; the Einstein model and Debye model for the heat capacity of phonon gas.

            \item Examples with quantum statistical physics: the harmonic oscillator; the quantum rotator; statistical analysis of the heat capacity of a diatomic gas.

            \item The grand canonical ensemble: chemical potential and the definition of the grand-canonical ensemble; the grand-canonical partition function and the grand-canonical potential; derivation of the Bose-Einstein and Fermi-Dirac distributions from the grand-canonical ensemble.

            \item Magnetism: coupling of spin to external magnetic fields; statistical derivation of Curie's law; application of polylogarithm functions to Fermi gases; the Ising model and the mean-field approximation; paramagnetic and ferromagnetic systems.

            \item Kinetic theory of gases: the Boltzmann distribution and its derivation from the canonical partition function; mean speed and mean free path of an ideal gas; derivation of the ideal gas equation from kinetic theory; diffusion and the diffusion constant of an ideal gas; viscosity and the coefficient of viscosity; analysis of thermal conductivity with kinetic theory.
        
        \end{itemize}
    \end{chapter}

\end{course}

\begin{course}{Physics Laboratory III}{https://www.fmf.uni-lj.si/en/study-physics/programmes/1fiz/2020/7000777/courses/1144/}{4}
    \label{physics_laboratory_3}

    \begin{chapter}{Experiments}
        \begin{itemize}
        
            \item Dynamics of the torsion pendulum

            \item Flexure of rods

            \item The Earth's magnetic field

            \item Heat conduction

            \item Magnetic field measurement with induction

            \item Imaging with diffraction lens

            \item The Michelson interferometer

            \item Absorption of gamma and beta radiation

            \item Properties of transmission lines and coaxial cables

            \item Piezoelectricity, 

            \item Characteristic of a silicon photodiode.        
        
        \end{itemize}
    \end{chapter}

\end{course}

\begin{course}{Physics Laboratory IV}{https://www.fmf.uni-lj.si/en/study-physics/programmes/1fiz/2020/7000777/courses/1145/}{4}
    \label{physics_laboratory_4}

    \begin{chapter}{Experiments}
        \begin{itemize}
            
            \item Acoustic resonator

            \item Measurement of the Boltzmann constant

            \item Ferroelectricity

            \item Ferromagnetism

            \item The Franck Hertz experiment

            \item The current-voltage characteristic of common electrical elements

            \item Coupled electronic oscillators

            \item The Millikan experiment and the elementary charge

            \item Diffraction of light

            \item Ultrasound imaging

            \item Precession of a spinning top
            
        \end{itemize}
    \end{chapter}
\end{course}

\begin{course}{Probability in Physics}{https://www.fmf.uni-lj.si/en/study-physics/programmes/1fiz/2020/7000777/courses/1177/}{3}
    \label{probability_in_physics}

    \begin{chapter}{Fundamentals of frequentist probability}
        \begin{itemize}
        
            \item Random experiments, outcomes, sample space, and events; random variables.

            \item Axioms of probability; certain, impossible, and disjoint events events; the inclusion-exclusion principle

            \item Conditional probability and independent events.
        
        \end{itemize}
    \end{chapter}

    \begin{chapter}{Probability distributions}

        \begin{itemize}
        
            \item The concept of discrete and continuous probability distributions; the probability density function and the cumulative distribution function.

            \item Discrete and continuous joint probability distributions; conditional probability; marginal distributions and independent random variables

            \item Normalization and transformation of distributions under and change of continuous variables.

            \item Definition, properties, and applications of important discrete probability distributions: the binomial and Poisson distributions

            \item Definition and properties of the normal distribution: probability density and cumulative distribution function; standardized normal distribution, the error function and complementary error function, standardized variables; standard deviation and full-width at half maximum; $ N $-$ \sigma $ intervals.

            \item Convolution: convolutions of independent distributions; expectation value, variance, and higher-order moments of convolved distributions.

            \item The central limit theorem.

            \item Definition, properties, and applications of other important continuous probability distributions: the uniform, exponential, Cauchy, and Maxwell distribution.

            \item Definition and properties of the Dirac delta distribution.

        \end{itemize}
    \end{chapter}

    \begin{chapter}{Quantifying distributions}
        \begin{itemize}
        
            \item Expectation value and variance of a continuous and discrete random variables and of functions of continuous and discrete random variables.

            \item Higher-order central and standardized moments of distributions; skewness and curtosis.

            \item Robust measures: median, median absolute deviation, and mode of a probability distribution.
            
            \item Quantiles and interquantile range.

            \item Expectation value in two dimensions; the covariance of random variables and its application to propagation of uncertainties; the coefficient of linear correlation; higher-order moments of bivariate distributions.

        \end{itemize}
    \end{chapter}

    \begin{chapter}{Introduction to statistics}
        \begin{itemize}
        
            \item Population and samples; degrees of freedom; sampling with and without replacement; population parameters and sample statistics, sample mean and sample variance.
            
            \item Estimators of population parameters; bias, consistency, and efficiency of an estimator;
            sample distributions of sums and differences; examples: parameter estimation of exponential and Gaussian distribution.

            \item Confidence intervals: confidence level, risk level, and confidence intervals; Student's $ t $ distribution, the $ t $ statistic, and computation of confidence intervals for the population mean; the chi-square distribution, the chi-square statistic, and computation of confidence intervals for the population variance.

            \item Linear regression: regression parameters, models, and the structure matrix; solution to the least squares problem by minimization of squared residuals; linearization of models; applications to parameter estimation and curve fitting.
        
        \end{itemize}
    \end{chapter}
\end{course}

\begin{course}{Numerical Methods}{https://www.fmf.uni-lj.si/en/study-physics/programmes/1fiz/2020/7000777/courses/524/}{3}
    \label{numerical_methods}

    \begin{chapter}{Introduction to numerical computation}
        \begin{itemize}

            \item Floating point representation of numbers and floating point arithmetic

            \item Sources of inexactness in numerical computation

            \item Sensitivity/conditioning of a problem, convergence of a method, and stability of computation; error analysis.

            \item Software for numerical computation.
            
        \end{itemize}
    \end{chapter}

    \begin{chapter}{Systems of linear equation}
        \begin{itemize}
            
            \item Matrix norms, determinants, and condition number.

            \item Upper and lower-triangular systems; Gaussian elimination; pivoting, LU decomposition without pivoting and with partial pivoting.

            \item Applications of LU decomposition: solving systems of linear equations, solving matrix equations, computing determinants, and computing matrix inverses.

            \item Methods of special linear systems: the Cholesky decomposition for positive-definite matrices; methods for tridiagonal systems.

            
        \end{itemize}
    \end{chapter}

    \begin{chapter}{Nonlinear equations}
        \begin{itemize}
        
            \item The bisection method

            \item Fixed-point iteration: convergence of iterative methods; Newton's method; the secant method.

            \item Roots of polynomial equations: the companion matrix; root reduction and the Laguerre method.

            \item Systems of nonlinear equations: fixed-point iteration and Newton's method.
        
        \end{itemize}
    \end{chapter}

    \begin{chapter}{Linear least square problems}
        \begin{itemize}
        
            \item General concepts: sample data points, parameters, model function, overdetermined systems, and residuals.

            \item Normal systems and the normal equations.

            \item QR decomposition: definition of the QR decomposition; (modified) Gram-Schmidt orthonormalization and application to QR decomposition; Givens rotations; Householder reflections.
        
        \end{itemize}
    \end{chapter}

    \begin{chapter}{The eigenvalue problem}
        \begin{itemize}
        
            \item Review from linear algebra: eigenvalues and eigenvectors; characteristic polynomial; Schur form and Schur decomposition.

            \item The power method for iteratively computing eigenvalues: algorithm, convergence, stopping criterion; reduction and Hotelling deflation for symmetric matrices; inverse iteration.

            \item QR iteration and convergence to the Schur form.

            \item Eigenvalues of symmetric matrices: tridiagonalization; Sturm sequences and the Sturm method; Jacobi iteration.
        
        \end{itemize}
    \end{chapter}

    \begin{chapter}{Polynomial interpolation}
        \begin{itemize}
        
            \item The polynomial interpolation problem: motivation; classic form of interpolation, the Vandermonde matrix and Vandermonde system.

            \item Lagrange interpolation: Lagrange polynomials; algorithm and error bound.

            \item Newton interpolation: divided differences, algorithm and error bound.
        
        \end{itemize}
    \end{chapter}

    \begin{chapter}{Numerical methods for ODEs}
        \begin{itemize}
        
            \item Methods for numerical integration: the Newton-Cotes rules; composite rules; Romberg extrapolation; Gaussian quadrature.

            \item Initial value problems: explicit Euler method; Runge-Kutta methods; single-step implicit methods; discussion of local and global error.

            \item Systems of first-order LDEs and initial value problems of higher order.

            \item Introduction to second-order boundary problems: general form; the finite difference method; the shooting method.
        
        \end{itemize}
    \end{chapter}

\end{course}

\begin{course}{Electronics I}{https://www.fmf.uni-lj.si/en/study-physics/programmes/1fiz/2020/7000777/courses/1134/}{3}
    \label{electronics_1}

    \begin{chapter}{Analog electric circuits}
        \begin{itemize}

            \item Introductory concepts: voltage and current; the concept of electrical ground; electrical power; Ohm's law; input and output impedance; Thevenin's theorem; Kirchhoff's rules.

            \item Definition, properties, and applications of common passive and active components: resistor, capacitor, inductor, diode, bipolar transistor, transformer.

            \item The operational amplifier: basic model of an op-amp using bipolar transistors; input and output current and voltage characteristics, gain, and frequency response; comparison of ideal and real op-amp properties.

            \item Operational amplifier circuits: comparator with and without hysteresis; voltage follower; inverting and non-inverting amplifiers; differential amplifier; instrumentation amplifier; sign inverter; absolute value circuit; logarithm and exponentiation circuit; differentiation and integration circuit; relaxation and Wien bridge oscillator.

        \end{itemize}
    \end{chapter}

    \begin{chapter}{Circuit analysis}
        \begin{itemize}
        
            \item Time-domain circuit analysis: differential equations, time derivative operator notation; the delta and unit step functions; a system's impulse and unit step response; the convolution sum and expansion over a delta function basis; examples: time-domain analysis of first-order $ RC $ high-pass and low-pass filters.

            \item Frequency domain analysis: the Laplace and Fourier transforms; frequency response and transfer function; poles and zeros of the transfer function; Bode diagrams.

            \item Filters: low-pass, high-pass, and band-pass filters; implementation with analog electronic elements; examples: first and second-order $ RC $ low-pass and high-pass filters, the second-order notch filter.

            \item Circuit stability: effect of transfer function poles on stability; cases of exponential decay, oscillation, and exponential growth; stability of feedback circuits: phase and gain margin; determining stability from Bode plots.
        \end{itemize}
    \end{chapter}

    \begin{chapter}{Modulation}
        \begin{itemize}
        
            \item Basic concepts: signal properties: amplitude, frequency, and phase; the concept of carrier and modulating signals; applications to information transmission.

            \item Amplitude modulation: analytical expressions for the carrier and modulating signals; the spectra of an amplitude-modulated signal; amplitude demodulation.

            \item Frequency modulation: analytical expressions for the carrier and modulating signals; frequency demodulation; examples of frequency demodulators: an $ LC $ circuit, averaging a comparator's square wave signal.

            \item Phase modulation: expressions for carrier and modulated signals; phase demodulation; spectrum of a phase-modulated signal.
        
        \end{itemize}
    \end{chapter}

    \begin{chapter}{Signal propagation along coaxial cables}
        \begin{itemize}
        
            \item Construction of the coaxial cable; cable parameters: impedance and inductance and capacitance per unit length.

            \item Solving the wave equation along a coaxial cable; speed of signal propagation in typical cables.

            \item Reflection and transmission: characteristic and terminating impedance; reflection and transmission at cable ends; impedance matching.
        
        \end{itemize}
    \end{chapter}
\end{course}

\begin{course}{Electronics II}{https://www.fmf.uni-lj.si/en/study-physics/programmes/1fiz/2020/7000777/courses/1135/}{3}
    \label{electronics_2}

    \begin{chapter}{Foundations of digital electronics}
        \begin{itemize}
        
            \item The binary and hexadecimal number systems; floating point representation of numbers; floating point arithmetic; binary representation of negative numbers.

            \item Hardware representation of binary numbers: conventional voltage ranges for logical one and logical zero; regulating voltage input into and output from digital circuits.

            \item Fundamental logic operations: the logical NOT, AND, and OR gates and the NAND, NOR, and XOR gates.

            \item Fundamentals of Boolean algebra: negation, conjunction, and disjunction; order of operations; associativity and commutativity of conjunction and disjunction; factoring; de Morgan's laws; identities for simplifying logic equations.

            \item Karnaugh maps for simplification of logical expressions.

            \item Addition and subtraction circuits: schematics of binary addition and subtraction; the half adder and full adder circuits; multi-bit adder; two's complement implementation of the subtraction circuit; combined addition-subtraction circuit.

            \item Multiplexers: definition and construction; combination of multiple multiplexers; applications to logic circuits; multiplexors as programmable logic devices.

            \item Selectors: definition and construction; use of an enable signal; combination of multiple selectors and applications to logic circuits.

            \item Precoder and decoder circuits and number encodings; comparison of Gray code and standard binary encoding.

            \item The monostable multivibrator: construction, circuit, and functioning mechanism; use in predictable-length pulse generation.

            \item Applications of the monostable multivibrator: pulse generation; double-pulse generator; frequency measurement circuit.
        
        \end{itemize}
    \end{chapter}

    \begin{chapter}{Flip-flops and registers}
        \begin{itemize}
        
            \item SR latches: construction, operating mechanism, and applications of the SR NOR, SR NAND, and gated SR latches.

            \item D latches: construction, operating mechanism, and applications of the D latch, clocked D latch, and master-slave D flip-flop.

            \item Construction, operating mechanism, and applications of the JK and T flip-flops.

            \item Registers: operating mechanism and construction using flip-flops; use for storage of multi-bit data.

            \item Shift registers: construction using D flip-flops; use for division-by-two and multiplication-by-two circuits.

            \item Basic principles of random-access memory and read-only memory.
        
        \end{itemize}
    \end{chapter}

    \begin{chapter}{Counters and finite-state machines}
        \begin{itemize}

            \item Transition tables and transition diagrams for design of logical circuits.
        
            \item Asynchronous counters: operating mechanism and construction using toggled flip-flops; counting to powers of two and non-powers of two; signal propagation through asynchronous counters and typical delays; use of asynchronous counters as frequency dividers.

            \item Synchronous counters: operating mechanism and construction using registers and a logic circuit; programmable counting methods using custom logic circuit implementation; signal propagation through asynchronous counters and typical delays; use of asynchronous counters as waveform generators.

            \item Mealy and Moore machines: definition; procedure for implementing Mealy and Moore machines: well-defined states, transition diagram, truth table, use of a default-state and handling of unused states; examples: variable upward and downward counter, debouncer circuit.
        
        \end{itemize}
    \end{chapter}

    \begin{chapter}{Analog-digital conversion}
        \begin{itemize}
        
            \item General considerations in digital-to-analog and analog-to-digital conversion: differential and integral non-linearity; resolution and effective number of bits; conversion time.

            \item Construction, operating mechanism, and use cases of common DACs: weighted-resistor inverting amplifier; $ R $-$ 2R $ ladder DAC; pulse-width modulation DAC.

            \item Construction, operating mechanism, and use cases of common ADCs: one-bit comparator; flash ADC; successive-approximation ADC; dual-slope DAC; charge-balancing ADC.

        \end{itemize}
    \end{chapter}

    \begin{chapter}{Fundamentals of electronic regulation}
        \begin{itemize}
        
            \item Linear regulators for steady output voltage supplies and their functioning mechanism using heat dissipation; example implementation of a low-voltage DC supply using \SI{220}{\volt} AC input.

            \item Switching regulators: duty cycle and its control with pulse-width modulated control signals; examples: buck step-down regulator, and boost step-up regulator, and flyback converters.

            \item Theory of proportional, differential, and integral regulation; transfer functions; effects of delay; oscillation and damping; example implementation of a PID temperature regulator in a fish tank.
        
        \end{itemize}
    \end{chapter}

    \begin{chapter}{Fundamentals of noise}
        \begin{itemize}
        
            \item Time-domain analysis of noise: noise as a stochastic process; quantification of noise with RMS voltage and power; energy and power in noisy signals; combination/superposition of multiple noise sources.

            \item Frequency-domain analysis of noise: review of Fourier analysis; the spectrum of a noisy signal and spectral power density; average power; the Parseval inequality; propagation of noise through transfer functions.

            \item Classification of noise: thermal (Johnson-Nyquist) noise, shot noise, pink noise, and popcorn noise; mechanisms for noise; optimizations for low-noise circuits.

            \item Fundamentals of noise reduction in multi-stage amplification and filter circuits.
        
        \end{itemize}
    \end{chapter}

\end{course}

\begin{course}{Electronics Laboratory}{https://www.fmf.uni-lj.si/en/study-physics/programmes/1fiz/2020/7000777/courses/1136/}{3}
    \label{electronics_laboratory}

    \begin{chapter}{List of labs and projects}
        \begin{itemize}
        
            \item Use of basic electronic instruments: multimeter, oscilloscope, function generator, and power supply.

            \item Current-voltage characteristic of the bipolar transistor

            \item Transistor implementation of a simple operational amplifier

            \item Frequency characteristics of the operational amplifier

            \item The instrumentation amplifier

            \item Op-amp integration and differentiation circuits.

            \item Relaxation oscillator

            \item A sample and hold circuit

            \item Amplitude detector

            \item Analog control circuits

            \item Switching power supply

            \item Frequency measurement circuit

            \item Logic gates

            \item Synchronous counter

            \item Monostable multivibrator

            \item Capacitance meter

            \item Analog multiplier

            \item Lock-in detection

            \item Voltage-to-frequency converter

            \item Amplitude modulation and demodulation

            \item Temperature regulation
        
        \end{itemize}
        
    \end{chapter}

\end{course}

\newpage
\section{Year 3}

\begin{course}{Electromagnetic Fields}{https://www.fmf.uni-lj.si/en/study-physics/programmes/1fiz/2020/7000777/courses/1133/}{7}
    \label{electromagnetic_fields}

    \begin{chapter}{Electrostatics}
        \begin{itemize}
            
            \item Coulomb's law and the electrostatic force.

            \item The electric field: definition in terms of the electrostatic force; electric field lines; electric circulation; electric flux; preservation of the electric field under orthogonal transformations.

            \item Electric potential and charge density: definition of the electric potential; the superposition principle for electrostatic force, field, and potential; electric charge density; examples: charge density of a point charge and electric dipole, surface and spherical charge distributions 

            \item The Poisson and Laplace equations for the electric potential; Green's functions for the Poisson equation in reciprocal space; general solution to the Poisson equation in term of Green's functions.

            \item Gauss's law: integral and differential forms.

            \item Electrostatic energy: energy of charge distribution in an external field; total electrostatic energy; electric field energy in terms of electric field.

            \item Electrostatic force and the stress-energy tensor.

            \item The multipole expansion in electrostatics: expansion of electric potential, electrostatic energy, force, and torque on a charge distribution in an external electric field; energy density of the electric field.

        \end{itemize}
    \end{chapter}

    \begin{chapter}{Magnetostatics}

        \begin{itemize}
            
            \item Ampere's force law and the magnetic force; examples: magnetic force between parallel wires and force between arbitrary current-carrying space curves.

            \item Fundamental quantities in magnetostatics: electric current density; magnetic field lines; magnetic circulation; magnetic flux.

            \item The Biot-Savart law for magnetic field; derivation of Ampere's law from the Biot-Savart law.

            \item The magnetic vector potential; relationship between magnetic vector potential and magnetic flux; the magnetic vector potential of an inductor; gauge transformations of the vector potential; the magnetostatic analog of the Poisson equation.

            \item Magnetic energy: energy of current distribution in an external magnetic field; total magnetostatic energy; magnetic field energy in terms of magnetic field; energy density of the magnetic field.

            \item Magnetic force and the magnetostatic stress-energy tensor.

            \item Multipole expansion of the magnetic field: the magnetic field of a magnetic dipole; Ampere equivalent for a circular current loop; multipole expansion of magnetic energy; tensor symmetrization; force and torque on a magnetic dipole in an external magnetic field.

            
        \end{itemize}
    \end{chapter}

    \begin{chapter}{Quasi-static electromagnetic fields}

        \begin{itemize}
        
            \item Electromagnetic induction: Lenz's law for magnetic flux; the quasi-static Maxwell equation for $ \curl \bm{E} $; Maxwell magnetic field impulse.

            \item The quasi-static Maxwell equations and the electromagnetic potentials for quasi-static electric and magnetic fields.

        \end{itemize}
    \end{chapter}

    \begin{chapter}{Conductors and Ohm's law}
        \begin{itemize}
        
            \item The Drude model of conductivity and the relaxation time constant of a conductor.

            \item Energy dissipation in a spatial charge distribution.

            \item Electric field energy and generalized, tensor form of capacitance for a conductor in space.

            \item Magnetic field energy and generalized, tensor form of inductance for a conductor in space.

            \item The skin effect in conductors carrying high-frequency alternating currents; electromagnetic field geometry and electric current density distribution in the skin effect.
        
        \end{itemize}
    \end{chapter}

    \begin{chapter}{Maxwell's equations in free space}
        \begin{itemize}
        
            \item Charge conservation, the continuity equation, and displacement current.

            \item Maxwell's equations in free space.

            \item Electromagnetic energy: the Poynting vector; energy dissipation; conservation of electromagnetic energy; the Poynting theorem in integral and differential forms.

            \item Electromagnetic momentum: definition of electromagnetic momentum; the Cauchy continuity equation for conservation of electromagnetic momentum in integral and differential form.

            \item Electromagnetic angular momentum: definition of electromagnetic angular momentum; the continuity equation for conservation of electromagnetic angular momentum in integral and differential form.

        \end{itemize}
    \end{chapter}

    \begin{chapter}{Electromagnetic fields in matter}
        \begin{itemize}
        
            \item Electric fields in matter: bound charge, electric polarization, and the $ \bm{D} $ field; the constitutive relation for electric fields and matter; properties of dielectric and conductors.

            \item Magnetic fields in matter: bound currents, magnetization, and the $ \bm{H} $ field; the constitutive relation for electric fields and matter; magnetization and magnetic dipole density; ferromagnetic, diamagnetic, and paramagnetic materials.
        
        \end{itemize}
    \end{chapter}

    \begin{chapter}{Frequency dependence of the dielectric function}

        \begin{itemize}
        
            \item General frequency dependence of the dielectric function $ \varepsilon(\omega) $ in the time and frequency domains.

            \item The Kramers-Kronig relations for the real and imaginary components of $ \varepsilon(\omega) $.

            \item Dissipation of electromagnetic energy and the imaginary component of the dielectric function

            \item Models for the dielectric function's frequency dependence: the classical equation of motion for bound charge and its solutions; Debye relaxation, Lorentz relaxation, and plasma relaxation; example: a phenomenological model of the dielectric function in water.
        
        \end{itemize}
    \end{chapter}

    \begin{chapter}{Introduction to the Hamiltonian formalism for electromagnetism}

        \begin{itemize}
        
            \item Review of the Lagrange and Hamilton equations from classical mechanics.

            \item The Lagrangian function of a charged particle in an electromagnetic field.

            \item The Hamiltonian function of a charged particle in an electromagnetic field.

            \item The Hamiltonian formalism for continuous charge distributions: the Schwarzschild invariant; the Lagrangian of for a continuous charge distribution in an electromagnetic field; the action of the electromagnetic Lagrangian; the Euler-Lagrange and Riemann-Lorenz equations for the electromagnetic potentials $ \phi $ and $ \bm{A} $.
        
        \end{itemize}
        
    \end{chapter}

    \begin{chapter}{Introduction to relativistic electromagnetism}
        \begin{itemize}
        
            \item Review of special relativity, four vectors, and Minkowski space.

            \item The Lorentz transformations of the electromagnetic field; invariance of Maxwell equations under Lorentz transformation.

            \item Electromagnetic quantities in Minkowski space: covariant and contravariant four vectors, scalar product of four-vectors, and the d'Alembert box operator; four vectors for electric current density and the electromagnetic potential; the covariant electromagnetic field tensor.
        
        \end{itemize}
    \end{chapter}

\end{course}

\begin{course}{Quantum Mechanics}{https://www.fmf.uni-lj.si/en/study-physics/programmes/1fiz/2020/7000777/courses/1156/}{7}
    \label{quantum_mechanics}

    \begin{chapter}{Fundamentals of quantum mechanics}
        \begin{itemize}
            \item The Schroedinger equation for a single particle in three dimensions. The wave function, probability density, and the continuity equation; properties of the wave function: normalization, continuity, differentiability; degeneracy and the nondegeneracy theorem.

            \item Stationary states of the Hamiltonian operator; expansion of wave functions over stationary states; time evolution of the wave function.

            \item Physical quantities as quantum operators; eigenvalues, eigenfunctions, and expectation values of quantum operators; the definition and properties of the momentum operator.

            \item The Ehrenfest theorem and its classical limit as Newton's second law; the virial theorem in quantum mechanics.

            
        \end{itemize}
    \end{chapter}

    \begin{chapter}{The Dirac formalism}
        \begin{itemize}

            \item Postulates of the Copenhagen interpretation of quantum mechanics.

            \item Review of Hilbert spaces; the braket notation for elements of Hilbert spaces, scalar products, and matrix elements.

            \item Expansion of wave functions and operators in an orthonormal basis; operator equations in an orthonormal basis.

            \item Definition and properties of Hermitian, adjoint, and self-adjoint operators.

            \item Definition and properties of orthogonal and unitary operators; unitary change of basis; generators of unitary transformations.

            \item The position (direct space) and momentum (reciprocal space) representations of quantum mechanics: review of Fourier analysis; eigenvalues and eigenstates of the momentum and position operators; expansion in the momentum eigenbasis; differentiation and multiplication in position and momentum space.

            \item Example: the quantum-harmonic oscillator: eigenvalues and eigenfunctions of the harmonic oscillator's Hamiltonian; the creation, annihilation, and counting operators; the ladder operators in matrix form; the generating formula for the harmonic oscillator's eigenfunctions in the coordinate representation; the harmonic oscillator in three dimensions; the coherent ground state.

        \end{itemize}
    \end{chapter}

    \begin{chapter}{Symmetries}
        
        \begin{itemize}
        
            \item Translational symmetry: the translation operator and one and three dimensions; problems with translation symmetry; relationship of translation symmetry to conservation of linear momentum; partial translational symmetry in a periodic potential.

            \item Rotational symmetry: the quantum operator for rotation by an angle $ \varphi $ about an arbitrary axis in space; infinitesimal and macroscopic rotations; problems with rotational symmetry; relationship of translation symmetry to conservation of angular momentum.

            \item Parity symmetry: definition and properties of the parity operator; relationship of the parity operator and problems with an even potential.

            \item Time reversal symmetry: definition and properties of the time reversal operator; the generalized time reversal and conjugation operator; generalized time reversal as an anti-unitary operator.

            \item Invariance under phase shift; change of a wave function and basis under a global translation in phase or potential energy; local and global gauge transformations; effect of gauge transformations on wave functions.
        
        \end{itemize}
    \end{chapter}

    \begin{chapter}{Angular momentum}
        \begin{itemize}
        
            \item Definition and properties of the angular momentum operator: Hermitian, cross product identities, angular momentum commutation relations.

            \item The angular momentum ladder operators: definition, commutation relations.

            \item Eigenvalues and eigenfunctions of the angular momentum operators $ L_{z} $ and $ L^{2} $: derivation using the ladder operators; magnetic and orbital quantum numbers; spherical harmonics; eigenbasis formed by the momentum eigenfunctions.

            \item Matrix representation of angular momentum: expansion of general wave functions in the angular momentum eigenbasis; matrix representations of $ L_{x} $, $ L_{y} $, $ L_{z} $, and $ L^{2} $.
        
        \end{itemize}
    \end{chapter}

    \begin{chapter}{The central potential}
        \begin{itemize}

            \item The general quantum-mechanics central potential problem: definition and properties of a central potential; the central potential Hamiltonian; transforming the Hamiltonian to spherical coordinates.

            \item The radial equation: derivation of the radial equation from the central potential Schroedinger equation; the effective potential; the radial eigenvalues and eigenfunctions; solutions to the radial equation in the limit $ r \to 0 $; free and bound states in the limit $ r \to \infty $.

            \item The Coulomb potential: general form of the Coulomb potential, Hamiltonian, and Schroedinger equation; the radial equation for the Coulomb potential; deriving energy eigenvalues with the Frobenius power-series method; degeneracy and eigenfunctions of the Coulomb Hamiltonian; application to the hydrogen atom.

            \item The semi-classical and classical limits: the Sommerfeld quantization condition; Coulomb potential eigenfunctions in the limit of large angular momentum; expectation values of  electron radius and squared radius; spherical shape of the orbit in the classical limit.
            
            
        \end{itemize}
    \end{chapter}

    \begin{chapter}{The charged particle in an electromagnetic field}

        \begin{itemize}
        
            \item The electrostatic potential and magnetic vector potential; the Coulomb gauge; derivation of the Hamiltonian and Schroedinger equation for a charged particle in an EM field under the Coulomb gauge.

            \item The normal Zeeman effect: magnetic dipole moment operator and the Bohr magneton; coupling of magnetic moment to a homogeneous external magnetic field; the Zeeman and quadratic coupling terms; the normal Zeeman effect.

            \item Landau levels: the Landau gauge potential; Landau quantization; application to a charged particle in planar cyclotron orbit; Landau levels; time evolution of Landau level eigenstates.

            \item Gauge transformations for particles in an electromagnetic field: local and global gauge transformations; application to time-dependent phase shifts of the wave function; gauge transformations of the electric and magnetic potentials; the Schroedinger equation under local gauge transformations.

            \item The Aharonov-Bohm effect: the Aharonov-Bohm experiment; the Schroedinger equation for a charged particle in the AB experiment; gauge transformations and analysis of the AB experiment; observable result: dependence of current branch ratio on magnetic flux.
        
        \end{itemize}
    \end{chapter}

    \begin{chapter}{Spin}
        \begin{itemize}
        
            \item Definition of spin; relationship to angular momentum; the spin operators; important properties of spin.

            \item Spin $ 1/2 $ particles: physical important of spin $ 1/2 $ particles; possible states of spin $ 1/2 $ particles; the $ \ket{\uparrow} $ and $ \ket{\downarrow} $ notation; matrix formulation of the spin operators for spin $ 1/2 $ particles.

            \item The Pauli spin matrix formalism: definition of the Pauli spin matrices $ \sigma_{x} $, $ \sigma_{y} $, and $ \sigma_{z} $; spin operators in terms of the spin matrices; commutation relations for the spin matrices; expectation values of the spin matrices.

            \item Spinors and spinor transformations: definition of spinors; spinors for spin $ 1/2 $ particles; transformation of spinors under rotation and time reversal; parameterization of spinor states using direction angles; transformation of the axis of quantization to arbitrary directions in space.

            \item Spin-orbit coupling: the spin magnetic moment; coupling of spin to an external magnetic field; derivation of the coupling term for spin-orbit coupling.

            \item The Stern-Gerlach experiment: experimental set-up; qualitative, semi-classical, and quantum-mechanical analysis of the experiment's outcome; implications for the quantization of magnetic moment.

            \item Addition of angular momentum for spin $ 1/2 $ particles: commutation of angular momentum; the outer product notation for multiplication of spin subspaces; shared eigenstates: the singlet and triplet states; Heisenberg coupling; the definition and use of the Clebsch-Gordan coefficients for addition of angular momentum.
        
        \end{itemize}
    \end{chapter}

    \begin{chapter}{Perturbation and approximation theory}
        \begin{itemize}
        
            \item The Rayleigh-Schroedinger method for first-order analysis for systems with non-degenerate spectra: motivation; Hamiltonian; first and second-order energy corrections; first-order formula for eigenvalues and eigenstates.

            \item First-order perturbative analysis for systems with degenerate spectra: ansatz for the Hamiltonian; the doubly-degenerate and $ N $-times degenerate cases; reduction to an $ N \times N $ eigenvalue problem.

            \item Systems with time-dependent Hamiltonians: ansatz for the Hamiltonian; the interaction picture and the general procedure for eigenfunctions; example: Rabi oscillations in a two-state system.

            \item Radiative transitions: transition probabilities and matrix elements under the influence of step-function potentials; first-order perturbative analysis; Fermi's golden rule.

            \item The WKB approximation: wave function ansatz; interpretation as the semiclassical limit $ \hbar \to 0 $; relationship to the Hamilton-Jacobi equation in the classical limit.

            \item The variational method for energy eigenvalues: motivation and applications; derivation of the energy eigenvalue formula.
        
        \end{itemize}
    \end{chapter}

    \begin{chapter}{Introduction to scattering}
        \begin{itemize}
        
            \item The scattering formalism in one dimension: role of the scattering potential; incoming and outgoing wave functions; expansion in plane wave states; probability current density in terms of probability amplitudes; the transfer matrix approach.

            \item The scattering matrix: definition and properties; relationship to the incoming and outgoing wave functions; parameterization in terms of probability amplitudes; the scattering matrix's determinant.

            \item Scattering in problems invariance under time-reversal and parity transformation.

            \item Scattering states and methods for normalization of plane waves:

            \item Overview of the scattering problem in three dimensions: expansion of wave functions in a basis of spherical waves; the scattering cross section and differential cross sections; the optical theorem.
        
        \end{itemize}
    \end{chapter}

\end{course}

\begin{course}{Optics}{https://www.fmf.uni-lj.si/en/study-physics/programmes/1fiz/2020/7000777/courses/1165/}{5}
    \label{optics}

    \begin{chapter}{Review of geometrical optics}
        \begin{itemize}
        
            \item Fermat's principle: the speed of light and the refractive index; Fermat's principle and minimization of optical action; derivation of the laws of reflection and refraction from Fermat's principle.

            \item The eikonal ray equation: statement and variational derivation of the ray equation from Fermat's least action principle; the ray equation in homogeneous matter and material with a parabolic refractive index; the ray equation in the limit of the paraxial approximation.

            \item Optical transfer matrices: the paraxial approximation; the transfer matrix method for propagation of light through optical interfaces; statement and derivation of transfer matrices for common optical elements.
        
        \end{itemize}
    \end{chapter}

    \begin{chapter}{Fundamentals of wave optics}
        \begin{itemize}
        
            \item Review of Maxwell's equations in isotropic, non-conducting matter; review of the wave equation in the context of electromagnetic waves.

            \item Plane waves: definition and properties of plane waves; plane waves as solutions to the electromagnetic wave equation; wave fronts and phase velocity; frequency, speed, and wavelength in matter.

            \item Direction and mutual orientations of the electric field, magnetic field, and wave vector for plane wave solutions in isotropic, non-conducting matter

            \item Ratio of field amplitudes and the impedance of free space.

            \item Electromagnetic power: energy density in electromagnetic waves; energy current density and the Poynting vector; electromagnetic power; Poynting's theorem.

            \item Polarization of plane wave solutions: mutual orientations of the electric field, magnetic field, and wave vector; quantities required to fully specify a plane wave; linear, circular, and elliptical polarization.

            \item The Jones formalism for polarization: Jones vectors and Jones matrices; statement and derivation of Jones matrices for common optical elements.

            \item The wave equation in conducting matter: free current density and Ohm's law; Maxwell's equations in conducting matter; generalization of the wave equation to conducting matter; the complex-valued refractive index; solutions to the wave equation in conducting matter; the imaginary refractive index and attenuation.
        
        \end{itemize}
    \end{chapter}

    \begin{chapter}{Reflection and refraction}
        \begin{itemize}
        
            \item Boundary conditions for Maxwell's equations in matter: statement and derivation of the boundary equations for the $ \vec{E} $, $ \vec{B} $, $ \vec{D} $, and $ \vec{H} $ fields.

            \item Electromagnetic wave behavior at the interface of optically distinct materials: conservation of phase, frequency, and tangential/normal field components; incident, reflected, and transmitted wave vectors and field components; derivation of the laws of reflection and refraction from wave principles.

            \item The Fresnel equations: decomposition of light into transverse-electric and transverse-magnetic polarizations; the Fresnel equations for TE and TM waves, transmission and reflection coefficients.

            \item Electromagnetic power in reflection and refraction: the Poynting vector at the interface of materials; transmittance and reflectance.

            \item Passage into optically denser material: dependence of the reflection and transmission coefficients on angle of incidence for TE and TM waves; Brewster's angle.
            
            \item Passage into optically less dense material: dependence of the reflection and transmission coefficients on angle of incidence for TE and TM waves; total internal reflection and the critical angle; the evanescent field; skin depth and attenuation; electromagnetic power in total internal reflection.

            \item Phase shift during reflection: relating transmission and reflection coefficients to phase shift for incident angles below the critical angle; phase shift in total internal reflection for TE and TM polarizations.

            \item Reflection from metals: complex-valued ansatz for the electric field; the complex-valued wave vector and refractive indices in metals; reflection of waves from metals; attenuation and skin depth in metals.
        
        \end{itemize}
    \end{chapter}

    \begin{chapter}{Diffraction}
        \begin{itemize}
        
            \item General concepts: definition of diffraction and its mechanism due to wave superposition; Huygen's principle; diffracting objects, aperture functions, and the optical axis; the Kirchhoff diffraction integral.

            \item Fraunhofer diffraction: derivation of the Fraunhofer diffraction integral from the large-field approximation to the Kirchhoff integral; the Fraunhofer diffraction integral as the Fourier transform of the aperture function; the Fresnel number and validity of the Fraunhofer approximation; examples: Fraunhofer diffraction on a thin slit, diffraction grating, rectangular aperture, and circular aperture.

            \item Fresnel diffraction: derivation of the Fresnel diffraction integral from the near-field approximation to the Kirchhoff integral; validity of the Fresnel approximation; Fresnel diffraction from a circular aperture; Fresnel zones and the Fresnel lens; Fresnel diffraction from a rectangular aperture and the Fresnel integrals.
        
        \end{itemize}
    \end{chapter}

    \begin{chapter}{Optical interference}
        \begin{itemize}
        
            \item General concepts: definition of interference and its mechanism due to wave superposition; superposition of plane waves; dependence of total electric field and energy flux density on phase difference between superposing waves; interferometric visibility.

            \item Interference via wavefront splitting: mechanisms for wavefront splitting; Young's double slit experiment and characteristics of the interference pattern; equivalence of Young's experiment and Fraunhofer diffraction.

            \item Interference via amplitude splitting: mechanisms for amplitude splitting; the Michelson interferometer and dependence of interference extrema on optical path difference; corrections for non-collimated beams.

            \item The Sagnac interferometer: mechanism and applications to fiber-optic gyroscopes.

            \item Thin-film interference: transmission and reflection coefficients for arbitrary incidence on a single thin films; construction of the Fabry-Perot interferometer and its applications in spectroscopy; matrix method for transmission and reflection coefficients for multiple thin films; application: anti-reflective coatings.

        \end{itemize}
    \end{chapter}

    \begin{chapter}{Optical scattering}
        \begin{itemize}
        
            \item General concepts: effect of three-dimensional objects on light propagation; scattering and extinction; the scattering cross section.

            \item Rayleigh scattering: validity of Rayleigh scattering; dipole scattering; the dipole antenna and Hertz radiation; dependence of radiated energy current density on angle and polarization; the scattering matrix and scattering efficiency.

            \item Mie scattering: validity of Rayleigh scattering; the scalar Helmholtz equation and separation of variables; solution for the electric field in terms of the associated Legendre polynomials and spherical Bessel functions; dependence of light intensity on wavelength of incident light and scatterer radius; transition between the regimes of Rayleigh scattering, Mie scattering, and geometrical optics.
        
        \end{itemize}
    \end{chapter}

    \begin{chapter}{Coherence}
        \begin{itemize}
        
            \item Fundamentals of temporal coherence: completely coherent, partially coherent, and incoherent waves; coherence time; microscopic mechanisms for lack of coherence; measuring temporal coherence with a Twynman-Green interferometer.

            \item Analysis of temporal coherence: expectation values of electric field and energy flux density; Fourier spectroscopy; the autocorrelation function and power spectral density; the Wiener-Khinchin theorem; the Van Cittert-Zernike theorem; examples: power spectral density and autocorrelation function of Gaussian and exponential pulses.

            \item Spatial coherence: coherence surface and definitions of spatial coherence; measuring spatial coherence with Young's double-slit experiment; example: estimating the size of starts using Young's double slit experiment and spatial coherence.
        
        \end{itemize}
    \end{chapter}

    \begin{chapter}{The refractive index}
        \begin{itemize}
        
            \item The Lorentz model of the refractive index: microscopic model of the refractive index: equation of motion, position, and electric dipole moment of a molecule; the electric displacement field and electric polarization.

            \item The complex refractive index and its real and imaginary components.

            \item The refractive index in the limit of a low-density gas; expressions for the refractive index with multiple resonances; the Cauchy and Sellmeier approximations of the refractive index.

            \item The refractive index in dense materials and the near-field correction for polarization and the total electric field.

            \item Introduction to optical metamaterials, negative refractive indices, and their applications.
        
        \end{itemize}
    \end{chapter}

    \begin{chapter}{Optical activity}
        \begin{itemize}
        
            \item General concepts: rotation of electromagnetic polarization on passage through material; right-hand circular and left-hand circular polarized light; dependence of refractive index on polarization.

            \item The Jones vector formalism for describing circular birefringence; dependence of phase difference on material thickness and RHC and LHC polarized refractive indices.

            \item The Faraday effect: equation of motion for an electron in an electromagnetic field; decomposition of linear polarization RHC and LHC components; LHC and RHC polarizations and refractive indices and the difference of refractive indices in the Faraday effect; the Verdet constant.

        \end{itemize}
    \end{chapter}

    \begin{chapter}{Optically anisotropic materials}
        \begin{itemize}
        
            \item General concepts: linear approximation to the constitutive relation; electric susceptibility and the dielectric constant; the tensor relationships between polarization and the displacement field and between the displacement and electric fields in anisotropic materials.

            \item Plane waves in anisotropic matter: Maxwell's equations and the wave equation in anisotropic matter; ansatzes and solutions for the electric and magnetic fields; eigenvalues of the dielectric tensor and the principle indices of refraction; directions and mutual orientations of the principle polarizations.

            \item The index ellipsoid formalism: quadratic form for electromagnetic energy density; normalized coordinates and construction of the index ellipsoid; use of the index ellipsoid for determining principle refractive indices and directions of polarization.

            \item Optically biaxial materials: solutions to the generalized wave equation in anisotropic matter; surface of constant wave vector; electric field solutions and polarizations in the $ k_{x} $, $ k_{y} $, and $ k_{z} $ planes; direction and orientation of the optic axis; direction and orientation of the Poynting vector; alternative formulations: surface of constant phase velocity and surface of constant ray speed.

            \item Optically uniaxial materials: ordinary and extraordinary polarizations; electric field solutions and polarizations and principle refractive indices in optically-uniaxial materials; directions and mutual orientations of the Poynting vector and the $ \vec{E} $ and $ \vec{D} $ fields; direction of the optic axis.

            \item Incidence on uniaxial materials: conservation of phase, frequency, and wave vector components on passage from air into uniaxial materials; reflection and refraction for TE and TM polarized light; shift of light in position space on transmission through uniaxial materials; special cases: normal incidence with optic axis tangent and normal to boundary plane.

            \item Optical components using birefringent materials: the Nicol prism polarizer; phase retarders and compensating planes, quarter waveplate and half waveplate; Glan-Taylor prism.
        
        \end{itemize}
    \end{chapter}

    \begin{chapter}{Introduction to lasers}
        \begin{itemize}
        
            \item Review of black body radiation: boson statistics for photons; the Bose-Einstein distribution; frequency distribution of black-body radiation; the Stefan-Boltzmann law.

            \item Interactions in an optical resonator: energy levels in a black-body cavity; probabilities of spontaneous emission, absorption, and stimulated emission; Fermi's golden rule and transition probabilities; the occupation equations; shape functions.

            \item Optical amplification: definition of and motivation for amplification; potential for amplification in the context of the occupation equations; amplification in a three-state system; inverted occupation and the condition for optical amplification; the stimulated emission cross section; low-power, high-power, and general amplification coefficients.

            \item Model of the laser: amplification module, energy inversion mechanism, and optical resonator; contributions to energy losses and condition for stationary state and stable functioning.
        
        \end{itemize}
    \end{chapter}
\end{course}

\begin{course}{Mechanics of Continuous Media}{https://www.fmf.uni-lj.si/en/study-physics/programmes/1fiz/2020/7000777/courses/1160/}{5}
    \label{mechanics_of_continuous_media}

    \begin{chapter}{Fundamentals of the theory of elasticity}
        \begin{itemize}
        
            \item Tensor formalism for deformations: concept of the continuum; the displacement vector; deriving the strain tensor; the strain tensor's symmetry and transformation under rotations.
            
            \item Physical meaning of the strain tensor's components: deformations along principle axes and change in angle between axes; bulk volume change the strain tensor's trace.
            
            \item Comparison of the Euler and Lagrange formulations of the strain tensor.

            \item Mechanical stress: assumption of short-range contact forces; derivation of the stress tensor; the stress tensor's symmetries; torque as a surface integral; examples: the stress tensor for isotropic, shear, and uniaxial deformations.
            
            \item The Cauchy equation and constitutive relation between the stress and strain tensors;
            thermodynamics of work and free energy in deformations.

            \item Elastic energy: characteristic polynomial and eigenvalues of the strain tensor in homogeneous, isotropic materials; expansion of free energy in the strain tensor's invariant quantities; the stress tensor in homogeneous, isotropic materials; the Lamé constants;
            relating the strain and stress tensor's trace.

            \item Hooke's law; the bulk, modulus, shear modulus, and inverse compressibility and their physical meaning;
            the Young modulus and the Poisson ratio and their physical meaning; relationships between the elastic constants.
            
            \item Derivation of the Navier equation from the Cauchy equation in homogeneous, isotropic materials;
            
            \item Symmetries, Hooke's law, and elastic moduli in crystals.

            \end{itemize}
    \end{chapter}

    \begin{chapter}{Applications of the theory of elasticity}
        \begin{itemize}

            \item Mechanics of plate flexure: definition of a plate; first-order expansion of the strain and stress tensors in plate-like geometry; the free energy, strain tensor, and stress tensor of a bent plate; plate equilibrium: variational condition for plate equilibrium, virtual work and displacements, variation of the plate's free energy.

            \item Lateral loads and plate buckling: definition of lateral loads; virtual work and displacements in plate-like geometry; strain tensor for plate flexure in lateral loads; work and critical load required for plate buckling; example: the critical lateral load on a rectangular plate.

            \item Mechanics of rod torsion: definition of a rod and rod torsion; first-order approximation of the displacement vector for rod torsion; the strain and stress tensors for rod torsion; the torsion function and the Poisson equation for torsion; equilibrium condition and the Laplace equation for torsion deformation; the associated torsion function and boundary conditions for torsion deformations; elastic energy in rod torsion and the elastic torsion constant.

            \item Mechanics of rod flexure: description of flexure in rod-like geometry, boundary conditions for a flexed rod; the strain and stress tensors for a flex rod; the geometrical profile of a bent rod; elastic energy in rod flexure and a rod's cross-sectional moment of inertia.

            \item Global rod theory: the tangent, normal, and binormal vectors for a rod; rod curvature and radius of curvature; arc-length parameterization of rods; rod rotation and flexure in terms of the tangent and normal vectors; a deformed rod's elastic energy and elastic constants; the equilibrium conditions for force and torque for a deformed rod; Kirchhoff's theory of rods.


            \item Elastic waves: transport of disturbances in elastic media; derivation of the wave equation from the Navier equation; wave equations and wave speeds for longitudinal and transverse wave polarizations; ratio of wave speeds and the relationship of the shear modulus and Poisson ratio to wave speeds.

            \item Elastic waves at an interface between materials: conservation of frequency, phase, and wave vector components; the laws of reflection and refraction for polarized elastic waves; the mixing of wave polarizations on passage through an interface.

            \item Surface waves: surface wave ansatz for the wave equation; boundary conditions for surface waves; solutions to the wave equation; attenuation and skin depth; transverse and longitudinal polarizations and their relative amplitudes for surface waves.
        
        \end{itemize}
    \end{chapter}

    \begin{chapter}{Mechanics of ideal fluids}
        \begin{itemize}
        
            \item The description of fluids: concept of the continuum, the fluid volume element, and the control volume; the field description of fluid state and the velocity, pressure, and density fields of a fluid.

            \item Conservation of mass and its implications: the mass of the fluid in control volume; volume flow rate, mass current density and volume current density through a surface or control volume; conservation of mass and the continuity equation; the continuity equation for incompressible fluids.

            \item The Euler equation: definition of an ideal fluid; Newton's law for a fluid element; the material derivative and the Euler picture of fluid mechanics; derivation of the Euler equation from Newton's law on a fluid element.

            \item Simplifications of the Euler equation: definition and thermodynamics of isentropic fluids, expression and simplification of the Euler equation for isentropic fluids in terms of enthalpy density; gradient-free form of the Euler equation;
            hydrostatics: simplification of the Euler equation to hydrostatic situations; thermodynamics of hydrostatic fluids and expression of the hydrostatic Euler equation in terms of Gibbs free energy density.

            \item Convection: definition of convection; condition for equilibrium under a temperature gradient; thermodynamics of convection: specific volume, entropy, heat capacities, and Maxwell's relations; condition for convection in fluids and in an ideal gas.

            \item The Bernoulli equation: definition and description of streamlines; description of the Euler equation in terms of enthalpy density; derivation of the Bernoulli equation from the Euler equation.

            \item Kelvin's circulation theorem: the circulation of a velocity field around a closed space curve; derivation of Kelvin's circulation theorem; vorticity.

            \item Potential flow: definition of potential flow; application of Kelvin's circulation theorem to potential flow; velocity potential description of the velocity field; simplifications of the Euler equation and Bernoulli equation for potential flow; incompressible potential flow.

            \item Stagnation points and the condition for analysis using incompressible flow.

            \item Complex formalism for two-dimensional, incompressible flow: the continuity equation; the Cauchy-Riemann equations for the velocity potential and stream function; area flow rate in terms of the stream function; use of conformal mappings for analyzing potential flow.

            \item Drag in potential flow: the dipole velocity field and its potential; decomposition of velocity field into homogeneous and dipole terms; the kinetic energy of potential dipole flow around obstacles; the induced mass tensor; d'Alembert's paradox.
        
        \end{itemize}
    \end{chapter}

    \begin{chapter}{Mechanics of viscous fluids}
        \begin{itemize}
            \item Construction and properties of the viscous stress tensor: proportionality to velocity gradient; symmetrization of the viscous tensor and vanishing for rotational flow; roles of the diagonal and off-diagonal elements; dynamical and volume viscosity; simplification of the viscous stress tensor for incompressible flow.

            \item Generalization of the Euler equation to viscous fluids; computing the divergence of the viscous stress tensor; the derivation of the Navier-Stokes equation.

            \item Kinetic viscosity and the velocity field for viscous fluids; the pressure field and Poisson equation for viscous fluids.

            \item Energy dissipation in incompressible, viscous fluids: kinetic energy and power of viscous fluid flow around obstacles; decomposition of power into volume transport and surface terms; expression for energy dissipation in terms of the velocity field.

            \item Hydrodynamic similarity: dimensionless form of the Navier-Stokes equation and characteristic dimension and velocity; the Reynold's number and the concept of hydrodynamic similarity.

            \item The Stokes approximation of the Navier equation for small Reynolds numbers; derivation of the Stokes formula for linear drag from small Reynolds number approximation of the Navier-Stokes equation: transformation to spherical coordinates, boundary conditions on the velocity field, the velocity and pressure fields, the force on an obstacle in Stoke's flow.

            \item Theory of the boundary layer: formation of the boundary layer; the Prandtl theory of the boundary layer; the Prandtl boundary layer equations and their transformation to dimensionless form; example: boundary layer for flow along a plate; the Blasius equation; backflow and separation of the boundary layer.

            \item Hydrodynamic instabilities: qualitative description of Rayleigh–Bénard instabilities and Taylor vortices; quantitative analysis of Kelvin-Helmholtz instabilities: the Euler equation and Bernoulli equation for two neighboring fluids; dynamics equations for the shape of the boundary surface; perturbations in the velocity potential; perturbative analysis of the velocity field for each fluid; the condition for stability in terms of densities, steady-state flow velocity, and wave vector.

            \item Introduction to turbulence: decomposition of the velocity field into steady-state and fluctuating terms; the Navier-Stokes equation in terms of steady and turbulent terms; time average of turbulent terms and correlation of velocity components; the Reynold's stress tensor for turbulent flow.
        \end{itemize}
    \end{chapter}
\end{course}

\begin{course}{Solid State Physics}{https://www.fmf.uni-lj.si/en/study-physics/programmes/1fiz/2020/7000777/courses/1138/}{7}
    \label{solid_state_physics}
    \begin{chapter}{Crystal structure}
        \begin{itemize}
        
            \item Description of crystal structure in position space: the Bravais lattice, primitive vectors, and the primitive unit cell; the basis and the description of physical crystals.

            \item Classes of unit cells: the parallelepiped, Wigner-Seitz, and conventional unit cells.

            \item Symmetries of Bravais lattices: translation, rotation, reflection, and rotation with reflection.

            \item Examples of Bravais lattices and crystal structures: the simple cubic, body-centered cubic, and face-centered cubic lattices; the Honeycomb and diamond structures; simple hexagonal and hexagonal close-packed lattices.

            \item Classification of Bravais lattices and crystal systems.

            \item The reciprocal lattice: review of the Fourier transform; definition of the reciprocal lattice, reciprocal lattice vectors, and reciprocal primitive vectors; the Wigner-Seitz cell in reciprocal space.

            \item Examples: reciprocal lattice of the simple cubic, face-centered cubic, body-centered cubic, and triangular Bravais lattices.

            \item Lattice planes: definition and construction of a lattice plane; families of lattice planes; Miller indices; relationship of the direct and reciprocal lattices.

            \item Scattering of x-rays: Bragg scattering and the Bragg interference condition; the Von Laue formulation of scattering and the Von Laue interference condition; equivalence of the Bragg and Von Laue formulations; the Ewald sphere formulation of scattering.

            \item Structure factors: the geometric structure factor and its relationship to crystal scattering; the GSF of common lattices; the atomic structure factor.

        \end{itemize}
    \end{chapter}

    \begin{chapter}{The Sommerfeld theory of free electrons}
        \begin{itemize}
        
            \item Foundational concepts: the Schroedinger equation for a free electron; energy eigenvalues and eigenfunctions of the Hamiltonian operator; periodic boundary conditions in crystals; the Fermi sphere and Fermi surface; the Fermi wave vector, energy, velocity, and volume.

            \item Important quantities in the Sommerfeld theory: the Fermi-Dirac distribution; the density of states and total number of states in a free electron gas; volume-normalized density of states; internal energy and heat capacity of a free electron gas.

            \item The Sommerfeld expansion: the Sommerfeld expansion of energy-dependent quantities near the Fermi energy; chemical potential, number of electrons, internal energy, and heat capacity of a free electron gas using the Sommerfeld expansion.
        
        \end{itemize}
    \end{chapter}

    \begin{chapter}{Electron states in a periodic potential}
        \begin{itemize}
        
            \item Bloch's theorem: periodic boundary conditions and periodicity of a crystal lattice; statement of Bloch's theorem; translation operators, commutation of the Hamiltonian and translation operators, and eigenvalues of the translation operator;
            derivation of Bloch's theorem using translation operators and the translational symmetry of crystal lattice; implications of Bloch's theorem on the quantization of reciprocal space.

            \item Expansion of lattice potential and electron wave function over Fourier modes; orthogonality of plane waves; energy bands and the restricted zone scheme; alternate derivation of Bloch's theorem by explicit construction;

            \item Generalized density of levels: implicit definition of the generalized density of levels; expansion of arbitrary energy-dependent quantities over the density of levels; density of levels in a fixed energy band and relationship to the gradient of the electron energy dispersion.

            \item The nearly-free electron model: expansion of Bloch states over plane waves and expansion of the lattice potential over Fourier modes; solutions to the Schroedinger equation in the NFEM in the limit of free electrons; solutions to the NFEM in a weak potential for non-degenerate and degenerate energy levels; relationship of degeneracy and the Bragg plane.

            \item The tight-binding model: review of the hydrogen atom problem and an atomic electron's eigenfunctions; construction of and motivation for the tight-binding model; expansion of Bloch states over atomic eigenfunctions; solutions to the Schroedinger equation in the tight-binding model; example: application of the tight-binding model to $ s $-orbital electrons.

            \item The semiclassical model of electron dynamics: review of free electron dynamics; postulates of the semiclassical model; the Bloch equations for the velocity and dynamics of Bloch electrons; conditions for validity of the semiclassical model; examples: dynamics of a Bloch electron in an external magnetic field and Bloch oscillations in an external electric field.

            \item Energy band occupation: condition for filled energy bands; electron current and heat current in filled bands; conditions that a material be an insulator or conductor; the concept of holes.

            \item The effective mass tensor: expansion of free electron dispersion about extrema; definition of the effective mass tensor; expression for the effective mass tensor in terms of Bloch electron velocity; effective mass tensor for holes; electron and hole dispersion near energy band minima.
        
        \end{itemize}
    \end{chapter}

    \begin{chapter}{Semiconductor physics}
        \begin{itemize}
        
            \item Foundational concepts: energy bands and energy gaps; valence band and conduction band; classification of conductors, insulators, and semiconductors by ground state energy band occupation configurations.

            \item Homogeneous semiconductors: definition of homogeneous semiconductors; the Fermi function and electron and hole number densities; the non-degenerate semiconductor approximation; energy dispersion relation near conduction band minima and valence band maxima; charge carrier number densities and densities of state in the non-degenerate semiconductor approximation.

            \item Intrinsic semiconductors: definition of intrinsic semiconductor; intrinsic charge carrier density; chemical potential and its temperature dependence in intrinsic semiconductors.

            \item Doped, homogeneous semiconductors: differences in effective mass, chemical potential, and energy levels between doped and intrinsic semiconductors; dopant energy levels; occupation statistics of dopant energy levels in the grand canonical ensemble; density of conduction band electrons and valence band holes in doped semiconductors; equilibrium charge carrier densities in n-type and p-type semiconductor; the Fermi energy and chemical potential in doped semiconductors; the limit cases of lightly and heavily-doped semiconductors.

            \item The p-n junction: coordinate system and geometry of the p-n junction; diffusion of charge carriers and the formation of the depletion region; energy levels and chemical potential across the p-n junction; relating energy band gap and the drop of electric potential; charge density, and the Poisson equation and its boundary conditions in the p-n junction; solutions for the electric potential and electric field across the p-n junction; width of the depletion region and the effects of external biasing;
            the p-n junctions current-voltage characteristic: electron and hole generation and recombination current and total electric current across the junction.
        
        \end{itemize}
    \end{chapter}

    \begin{chapter}{Lattice oscillations}
        \begin{itemize}
        
            \item The classical theory of lattice oscillations: equations of motion, their solution, dispersion, and propagation speeds in a one-dimensional monoatomic lattice and a one-dimensional lattice with a diatomic basis; the acoustic and optical branches for lattices with a basis; generalization of the one-dimensional analysis to three-dimensional monoatomic lattices and lattices with a basis.

            \item The quantum theory of lattice oscillations: the lattice Hamiltonian; Fourier expansions of position and momentum; commutation relations of the position momentum operators and their Fourier transforms; expansion of the lattice Hamiltonian in Fourier modes; review of the quantum harmonic oscillator: Hamiltonian, energy levels, and the ladder operator formalism; decomposition of the lattice Hamiltonian into a sum of uncoupled harmonic oscillators.

            \item Applications of quantum lattice oscillations: review of the canonical partition function, free energy, and internal energy from quantum statistical physics; the phonon contributions to the internal energy and free energy of a harmonic lattice.

            \item The phonon contribution to specific heat: general expression for a harmonic lattice's internal energy and specific heat; low-temperature and high-temperature limits of the specific heat; the Einstein model for the specific heat of a harmonic lattice; the Debye model: approximation of lattice dispersion with linear acoustic branches, Debye temperature, energy, and wave vector; general Debye approximation for lattice specific heat and its high and low-temperature limits.

            \item Introduction to anharmonic effects: thermal expansion in a 1D lattice; the kinetic theory of phonon heat conduction; phonon transitions, normal processes and umklapp processes, and low and high-temperature limits of phonon mean free time; the temperature dependence of a lattice's thermal conductivity.
        
        \end{itemize}
    \end{chapter}
\end{course}

\begin{course}{Measurement Techniques}{https://www.fmf.uni-lj.si/en/study-physics/programmes/1fiz/2020/7000777/courses/1441/}{6}
    \label{measurement_techniques}

    \begin{chapter}{Error analysis and optimal combination}
        \begin{itemize}
        
            \item Review of continuous and discrete random variables, probability densities, and cumulative distribution functions; expected value, variance, and higher moments of discrete and continuous random variables.

            \item Jointly-distributed random variables: joint probability density and cumulative distribution functions; variance, covariance, and correlation of jointly-distributed random variables.

            \item Error analysis: systematic and random errors; propagation of uncertainty; the central limit theorem and approximation of random variables with the normal distribution; combination of measurements with different uncertainties.

            \item Minimum-variance combinations of independent normally-distributed random variables and minimization of the chi-squared quadratic form; correlation coefficients, covariance, and minimum variance combinations of dependent normally-distributed random variables; variance of the mean.

        \end{itemize}
    \end{chapter}

    \begin{chapter}{The Kalman filter}
        \begin{itemize}
        
            \item Optimal tracking of a constant scalar quantity: optimal estimate and its variance for a repeatedly-measured scalar constant; discrete recursive formula for online minimum-variance estimation; continuous-time approximation of variance and the differential equations for the optimal estimate and its variance.

            \item Optimal tracking of a variable scalar quantity: dynamical equation for scalar variable and its difference equation approximation; process noise; prediction of a scalar variable from the dynamical equation under the influence process noise; minimum-variance update of the extrapolated estimate; transition to the continuous-time formulation; discrete and continuous recursive formulae for online minimum-variance estimate and its variance for a scalar variable.

            \item The Kalman filter for a vector variable quantity: dynamical equation for a vector variable and its difference equation approximation; observation noise and process noise and the time dependence of their covariance matrices; the predict and update steps for a vector variable; transition to the continuous-time formulation; discrete and continuous Kalman filter algorithms for online, optimal tracking of a random variable. 
        
        \end{itemize}
    \end{chapter}

    \begin{chapter}{Analysis of sensors and filters}
        \begin{itemize}
        
            \item Definition and properties of the Laplace transform and its application to frequency-domain analysis of linear, constant coefficient differential equations.

            \item Time-domain analysis of sensors: formulation of sensors in terms of constant-coefficient, linear differential equations; the order of a sensor; the impulse response of a sensor; impulse response and response to characteristic inputs (e.g. linear and quadratic inputs, the unit step, etc...) for first and second-order sensors.

            \item Frequency-domain analysis of sensors: the Laplace transform of a sensor's differential equation; the frequency response and transfer function of a sensor; phase shift and amplitude response of a transfer function; the response of systems to periodic input; Bode plots.

            \item Impact of sensors on a measured system: Thevenin's theorem; input and output impedance; the use of operational and instrumentation amplifiers as high input impedance buffers; transmission of signals by cables: impedance matching and the characteristic impedance of a cable.

            \item Thermal noise: classification of noise (thermal (Johnson-Nyquist) noise, shot noise, pink noise, and popcorn noise), RMS voltage and power of thermal noise; the Wiener-Khinchin theorem and the power spectral density of thermal noise on a resistor; quantum-statistical interpretation of thermal noise and its high-frequency power spectral density; propagation of thermal noise through linear circuits.
        
        \end{itemize}
    \end{chapter}

    \begin{chapter}{Statistics}
        \begin{itemize}
        
            \item Estimation of sample statistics: populations and samples; the sample mean and standard deviation; biased and unbiased estimators, maximum-likelihood estimation; estimators of the sample mean and variance; the chi-square distribution and Student's $ t $ distribution.

            \item Confidence intervals and hypothesis testing; testing of the sample mean with Student's $ t $ test and testing of the sample variance with the chi-square test.

            \item Goodness-of-fit tests: Pearson's Chi-Squared test; Fischer's test and the Fischer statistic; the Kolmogorov-Smirnov test.

            \item Linear least squares: parameters, models, and the structure matrix; solution to the least squares problem by direct minimization of squared residuals and by using the Kalman filter formalism; linear least squares for measurements with difference variances; application of linear least squares to signal peak detection.
        
        \end{itemize}
    \end{chapter}

    \begin{chapter}{Important measurement techniques}
        \begin{itemize}
        
            \item Negative feedback systems: principles of negative feedback; application of negative feedback to increasing a control system's stability and accuracy; gain, phase, and damping criteria for stable functioning of negative feedback systems.

            \item Lock-in detection: working principles and instrumental implementation; application to detection of periodically-modulated quantities in noisy backgrounds; examples: Auger electron spectroscopy, electron spin resonance, nuclear-magnetic resonance.

            \item Phase-locked loops: working principles and instrumental implementation of a phase-locked loop: phase detector, regulatory filter, and voltage-controlled oscillator; type I and type II phase detectors; choice of transfer function for the regulatory filter; transfer function of the PLL; stability and locking dynamics of a PLL; applications of PLL to frequency synthesis and division and to radio transmission.

            \item Displacement sensors: working principles and instrumental implementation of resistive, capacitive, piezoelectric, and inductive displacement sensors.

            \item Review of measurement techniques for frequency, time, temperature, acceleration, and angular velocity.
        
        \end{itemize}
    \end{chapter}
\end{course}

\begin{course}{Measurement of Ionizing Radiation}{https://www.fmf.uni-lj.si/en/study-physics/programmes/1fiz/2020/7000777/courses/1446/}{5}
    \label{measurement_of_ionizing_radiation}

    \begin{chapter}{Interaction of radiation with matter}
        \begin{itemize}
        
            \item Overview of interaction processes: inelastic collisions of ionizing particles with electrons, elastic collisions with nuclei, Cherenkov radiation, brehmsstrahlung processes, and nuclear reactions.

            \item Interaction of charged particles with matter: semiclassical analysis of ionizing energy losses; mean excitation potential; relativistic corrections to energy losses at high speeds and quantum corrections for electron spin; the Bethe-Bloch formula for ionizing energy losses; range of an ionizing particle in matter.

            \item Interaction of photons with matter: the scattering cross section and exponential attenuation of photons in matter; the photoelectric effect; Compton scattering and the Klein-Nishina formula for the Compton scattering cross section; pair production; energy dependence of photon-matter interactions.
        
        \end{itemize}
    \end{chapter}

    \begin{chapter}{Particle detectors}
        \begin{itemize}
            
            \item Accelerator systems: ion sources; principles of linear, cyclotron, and synchrotron accelerators; transport of charged particles in accelerators.

            \item Cylindrical gas-based detectors: review of the electric field and electric potential in cylindrical geometry; construction and working principles of cylindrical gas-based detectors; mechanisms of ionization; working regimes, the multiplication factor, secondary ionization, and Geiger-Muller counters; energy resolution and the Fano factor; transport of electrons and ions in gas; dynamics of the electric signal in cylindrical gas-based detectors; shaping and filtering of the detector signal.

            \item Position-sensitive detectors: multi-wire proportional chambers (construction and working principles; position resolution; wire sagging); drift chambers (construction and working principles; transport of ions in gas; position resolution).

            \item Semiconducting detectors: charge density, electric potential, and electric field in diffused p-n junction diodes; width and response of diffused-junction diodes to external biasing; the photoelectric effect, built-in potential, and working principles of semiconducting detectors; signal dynamics in diffused junction diode detectors; review of common semiconducting detectors: diffused-junction diodes and PIN diodes, surface barrier detectors; Schottky diodes; compensated semiconductor detectors; silicon-lithium and germanium-lithium detectors.

            \item Scintillating detectors: quantum energy levels and the mechanisms for fluorescence and phosphorescence in organic and inorganic scintillators; review of the advantages and disadvantages of organic and inorganic scintillators; time evolution of scintillation photon emission, the photoelectric effect, and the working principles of scintillating detectors; electric signal dynamics in a scintillating detector.

        \end{itemize}
    \end{chapter}

    \begin{chapter}{Particle identification}
        \begin{itemize}
        
            \item Particle classification: review of the standard model; mass and charge of common particles in high-energy physics experiments.

            \item Multiple measurements of ionizing losses and their application to particle mass measurement and particle identification.

            \item Time-of-flight measurement: construction and working principles of time-of-flight detectors and their application to particle mass measurement.

            \item Cherenkov detectors: Cherenkov radiation, the Cherenkov angle, and the spectral distribution of Cherenkov photon frequency and wavelength; construction and working principles of threshold and ring-imaging Cherenkov detectors and their application to particle mass measurement and particle identification.

            \item Neutron detection: detection of thermal neutrons with shielded gas-based detectors; post-scattering energy distribution of fast neutrons; slowing fast neutrons to thermal energy and detection of fast neutrons.

            \item Photon identification: the photoelectric effect; construction and working principles of photomultiplier tubes; efficiency of photomultiplier tubes; semiconducting photon detectors: microchannel plates, avalanche photodiodes, and hybrid avalanche photodiodes.
        
        \end{itemize}
    \end{chapter}

    \begin{chapter}{Radiation safety}
        \begin{itemize}
        
            \item Review of radioactivity: activity, decay time, radiation statistics; alpha, beta and gamma radiation; isotopes and radioactive series; sources of radiation (natural and man-made isotopes, cosmic radiation).

            \item Quantities involved in radiation dosimetry: radioactive activity and decay constants; absorbed dose and dose rate; equivalent dose; effective dose.

            \item Review of deterministic and stochastic biological consequences of radiation exposure.

            \item Legal regulations of radiation exposure for specialized occupations and for the general civilian population.

        \end{itemize}
    \end{chapter}
\end{course}

\begin{course}{Mathematical Physics Lab}{https://www.fmf.uni-lj.si/en/study-physics/programmes/1fiz/2020/7000777/courses/1159/}{6}
    \label{mathematical_physics_lab}
    
    \begin{chapter}{Projects}
        \begin{enumerate}
        
            \item Algorithmic implementation of the Airy functions $ \mathrm{Ai}(x) $ and $ \mathrm{Bi}(x) $ with an absolute and relative error of less than $ 10^{-10} $ on the entire real line; implementation of an algorithmic routine for computing the Airy functions' first 100 zeros.

            Skills: numerical implementation of the Airy function's Maclaurin and Taylor series and asymptotic series expansions; Padé approximants; numerical precision, the limitations of floating-point arithmetic, and the use of arbitrary-precision libraries; error evaluation; root-finding algorithms.

            \item Random walks in two and three-dimensions and their application to diffusion; numerically computing diffusion constants and classifying random walks into normal, anomalous, and ballistic diffusion regimes.

            Skills: random number generation and random seeding; distribution-generating functions and sampling from distributions; Monte-Carlo methods; computation of Lévy flights and Lévy walks; computation of mean squared displacement and median absolute deviation; regression estimation of random walk parameters.

            \item The eigenvalue problem: numerically computing the eigenvalues, eigenvectors, and eigenfunctions of a perturbed quantum harmonic oscillator and of a particle in a double-well potential.

            Skills: efficient construction and computation of perturbation matrix elements;
            implementation of numerical methods for computation of the eigenvalues of general and symmetric matrices: power iteration, QR decomposition via Givens rotations, Householder reflections, the Jacobi method for symmetric matrices, tridiagonalization schemes; stopping conditions for iterative methods;
            computation of eigenvectors and construction of eigenfunctions using eigenvectors as coefficients of an orthonormal basis expansion;
            analysis of accuracy and computation time for eigenvalue methods; 
            practical experience with perturbed Hamiltonians in quantum mechanics.

            \item The discrete Fourier transform, sampling, and spectral analysis.

            Skills: numerical computation of the DFT and inverse DFT (in $ \mathcal{O}(N^{2}) $ time using the Fourier transform matrix; the FFT follows in the next project); familiarity with sampling, aliasing, spectral leakage, frequency shifting, zero-padding, and spectral peak detection; computation time analysis.
            Spectral analysis of audio signals (e.g. detecting frequency and names of guitar notes and classifying modes of an acoustic resonator)
        
            \item The fast Fourier transform and its application to efficient computation of convolution, cross-correlation, and auto-correlation; spectral analysis.

            Skills: numerical computation of the FFT and inverse FFT in $ \mathcal{O}(N \log N) $ time; linear and circular convolution and efficient implementation of linear convolution with the FFT; cross-correlation and auto-correlation and their computation with the FFT; auto-correlation and spectral methods for detection of periodic signals embedded in a noisy background.

            \item The first-order initial value problem and its application to the temperature distribution on a rod.

            Skills: discretization of time and position; explicit and implicit numerical methods for linear differential equations (including implementation of the explicit and implicit Euler; the trapezoid method; Heun's method; the explicit and implicit midpoint method; the RK2 and RK4 Runge-Kutta methods; Ralston's 3rd-order RK3 method; the 3/8 4-th order RK method); embedded, adaptive step methods (including implementation of the 2nd order Bogacki-Shampine method used in Matlab's \texttt{ode23}; the 4-5th order Runge-Kutta-Fehlberg method; the 4-5th order Cash-Karp method; the 4-5th order Dorman-Prince method used in Matlab's \texttt{ode45}); the multi-step Adams-Bashforth-Moulton predictor-corrector method; error analysis; computation time analysis; stability analysis.

            \item The non-linear second-order initial value problem and its application to the dynamics of the non-linear mathematical pendulum, a driven, damped mathematical pendulum, and the Van Der Pol oscillator.

            Skills: numerical solutions of higher-order and vector-valued ordinary differential equations;
            implementation of symplectic integration methods (the 2nd-order Verlet method and the 4-th order position-extended Forest-Ruth method)
            implementation of standard numericla methods for ODEs (explicit Euler, Heun's method, the 2nd-order Runge-Kutta midpoint method, Ralston's 3rd-order RK3 method, 3rd-order strong, stability-preserving RK method, the classic and Ralston version of RK4, adaptive-step Runge-Kutta-Fehlberg and Cash-Karp methods, the 4th-order multistep predictor-corrector Adams-Bashforth-Moulton method)
            computation and plotting of phase spaces and vector fields
            numerical computation of elliptic integrals with arbitrary-precision libraries;
            analysis of solution accuracy and algorithmic efficiency.

            \item The second-order boundary value problem in the context of second-order differential operators and their eigenvalues and eigenfunctions.
            Application to quantum mechanics: numerically computing the energy eigenvalues and eigenfunctions of a particle in an infinite and finite potential well.

            Skills: implementation of the finite difference method and shooting method for second-order boundary value problems; discretization of position space; numerically computing the eigenvalues and eigenfunctions of a differential operator; error analysis of the finite difference method.

            \item Spectral methods for partial differential equations and their application to solving the heat equation for time evolution of the temperature distribution on a rod for various initial conditions and boundary conditions.

            Skills: implementation of the Fourier spectral method for partial differential equations and the collocation method using cubic B-splines;
            familiarity with periodic boundary conditions and Dirichlet boundary conditions;
            analysis of computational efficiency.

            \item The Crank-Nicolson method and its higher-order generalizations for partial differential equations, and their application to quantum mechanics in solving the Schroedinger equation for time evolution of a coherent Gaussian initial state in a harmonic potential and for the time evolution of a Gaussian wave packet propagating through free space.

            Skills: numerical implementation of the Crank-Nicolson method for PDEs; unitary approximations of the Hamiltonian operator;
            Padé approximation of the exponential function and implementation of higher-order position and time generalizations of the Crank-Nicolson method (a reproduction of the results in W. van Dijk and F. M. Toyama. ``Accurate numerical solutions of the time-dependent Schrödinger equation.'' Phys. Rev. E \textbf{75}, 036707 (2007).);
            analysis of computation time and solution accuracy.

            \item The Galerkin method for partial differential equations and its application to solving for the velocity profile and Poiseuille coefficient of laminar flow in a pipe with a semi-circular cross section and to solving the first-order wave equation on a string.

            Skills: numerical implementation of the Galerkin method for PDEs and its applications to computational fluid dynamics; computation of Galerkin basis functions and Bessel functions; extensive use of vectorization for efficient implementation of operations on arrays; analysis of solution error and computation time.

            \item Feed-forward deep neural networks (aka multilayer perceptrons aka fully-connected neural networks) for classification problems and their application to classification of collision products in high-energy particle physics (a reproduction of the results in Baldi, P., P. Sadowski, and D. Whiteson. “Searching for Exotic Particles in High-Energy Physics with Deep Learning.” Nature Communications 5 (July 2, 2014).).

            Skills: feature engineering; data normalization and scaling; the problem of under-fitting and over-fitting; practical experience with common loss functions and optimization algorithms; evaluating classifiers with confusion matrices and receiver-operating characteristic curves; regularization and early stopping for prevention of over-training.
        
        \end{enumerate}
    \end{chapter}
\end{course}

\begin{course}{Collecting and Processing Data}{https://www.fmf.uni-lj.si/en/study-physics/programmes/1fiz/2020/7000777/courses/1178/}{3}
    \label{collecting_and_processing_data}

    \begin{chapter}{Model of the processor}
        \begin{itemize}
        
            \item Digital representation of numbers: the binary and hexadecimal number systems; common number formats: 8-bit and 16-bit integer, 32-bit and 64-bit floating point; precision in floating point arithmetic.

            \item High and low logic levels in digital circuits; power supply and grounding of sensors;
            non-referenced single-ended, referenced single-ended, and differential measurement systems.

            \item Properties of operational and instrumentation amplifiers; the input and output impedance of an amplifier; application of amplifier stages to measurement systems.

            \item Buses: motivation for the use of buses for data transfer; three-state switches and buss access; read and write signals, selectors, and the read-write cycle; time delays in data transfer through buses: typical orders of magnitude, glitches, and accounting for time delays.

            \item Types of buses: parallel buses and the parallel bus interface; serial bus and the serial peripheral interface; the universal serial bus (USB) and peripheral component interconnect (PCI).

            \item Components of the processor: the arithmetic logic unit; registers; status register and the FLAGS register; the control unit; read-only memory (ROM) and random-access memory (RAM); operations in the central processing unit.

        \end{itemize}
    \end{chapter}

    \begin{chapter}{Analog-digital conversion}
        (Material like in Electronics 2)
        \begin{itemize}
        
            \item General considerations in digital-to-analog and analog-to-digital conversion: differential and integral non-linearity; resolution, effective number of bits, accuracy, and conversion time.

            \item Construction, operating mechanism, and use cases of common DACs: weighted-resistor inverting amplifier; $ R $-$ 2R $ ladder DAC; pulse-width modulation DAC; the selection and use of a DAC in practice.

            \item Construction, operating mechanism, and use cases of common ADCs: one-bit comparator; flash ADC; successive-approximation ADC; dual-slope ADC; charge-balancing ADC; the selection and use of a ADC in practice.

        \end{itemize}
    \end{chapter}

    \begin{chapter}{Data acquisition}
        \begin{itemize}

            \item Sampling: sample rate; spectrum and bandwidth of an analog input signal; the Nyquist criteria and the sampling theorem.

            \item Decimation and interpolation, i.e. downsampling and upsampling.

            \item Aliasing; pre-processing and filtering before sampling; impedance matching; band pass of the measurement system.

            \item Working principles and practical use of common sensors: temperature, light, displacement, sound, acceleration, pressure, mechanical strain.
            
        \end{itemize}
    \end{chapter}

    \begin{chapter}{Digital signal processing}
        \begin{itemize}

            \item Discrete signals and systems in the time domain: important discrete signals and expansion of an arbitrary discrete signal over a basis of unit impulses; definition and properties of linear, shift-invariant systems; convolution, impulse response, and the convolution sum for an LSI system; discrete-time systems described by linear, constant-coefficient difference equations.

            \item The frequency response of a discrete system; frequency response of LSI systems to complex exponential and sinusoidal inputs; amplitude and phase response of a discrete-time system.

            \item The discrete-time Fourier transform: definition and properties of the DTFT; relationship of the continuous and discrete-time Fourier transforms.
            
            \item The Z-transform: definition and properties of the Z-transform; the region of convergence; relationship of the Z-transform and DTFT.

            \item The discrete Fourier transform: periodic and finite-length sequences; the discrete Fourier series of a periodic signal; properties of the Fourier series; definition and properties of the discrete Fourier transform; circular convolution and the implementation of linear convolution using the DFT.

            \item Filtering: impulse and frequency response of low-pass, high-pass, band-pass, band-stop filters, and all-pass filters; filter transition band and stopband; frequency-domain analysis of filters; time domain implementation of filtering with convolution;
            the overlap-add method and real-time implementation of filters.
            
            \item FIR filtering: properties of FIR filters; window functions and the window method for FIR filter design; the least-mean-square error method; FIR implementation of Hilbert transformers.

            \item IIR digital filtering: transfer function and properties of an IIR filter; poles and stability of IIR filters; impulse-invariance, matched Z-transform, and bilinear transform methods for IIR filter design;
            choosing an filter in practice, comparison of the advantages and disadvantages of FIR and IIR filters.

            \item Amplitude, frequency, and phase modulation and demodulation.
        
        \end{itemize}
    \end{chapter}
\end{course}

\begin{course}{Physics Laboratory V}{https://www.fmf.uni-lj.si/en/study-physics/programmes/1fiz/2020/7000777/courses/1146/}{4}
    \label{physics_laboratory_5}

    \begin{chapter}{Experiments}
        \begin{itemize}
            
            \item Electronic spin resonance

            \item The diffusion of liquids

            \item Holography

            \item The electro-optical effect

            \item Propagation of microwaves in waveguides

            \item Nuclear magnetic resonance

            \item X-rays

            \item Gamma ray spectroscopy

            \item Angular correlation of annihilation gamma rays

            \item The Hall effect
            
        \end{itemize}
    \end{chapter}
\end{course}

\begin{course}{Industrial Physics}{https://www.fmf.uni-lj.si/en/study-physics/programmes/1fiz/2020/7000777/courses/1149/}{3}
    \label{industrial_physics}

    \begin{chapter}{Material}
        \begin{itemize}
        
            \item The innovation process, intellectual property, and patents.

            \item Introduction to medical physics: X-ray imaging, magnetic resonance imaging, ultrasound, nuclear medicine, radiotherapy.

            \item Building engineering physics: acoustics in rooms, thermal insulation, heating, water vapour transport, illumination in rooms.

            \item Introduction to nanotechnology: microscopy, nanomaterials, devices at the nanoscale. 

            \item Liquid-crystal displays: principles and applications.

            \item Photonic crystals: synthesis and applications.

            \item Miscellaneous topics: photovoltaics; Superconductivity; Non-invasive investigations of materials and constructions.
        
        \end{itemize}
    \end{chapter}
\end{course}

\begin{course}{Seminar}{https://www.fmf.uni-lj.si/en/study-physics/programmes/1fiz/2020/7000777/courses/1173/}{3}
    \label{seminar}

    \begin{chapter}{Course description}
        Students, under the guidance of a faculty mentor, prepare a short undergraduate thesis and presentation on a theme from physics and related interdisciplinary fields.
        The subjects are suggested by a mentor or chosen by the student, preferably from newer fields not yet incorporated into the undergraduate curriculum.
        The presentation is oral in Slovene or English, after submission of an accepted written text and presentation graphics.

        The course is intended primarily as an exercise in clear scientific writing and presentation---a training, in some sense, for writing papers and giving presentations at scientific conferences.
        However, students are neither expected nor encouraged to produce original research in the scope of the Seminar course.

    \end{chapter}
\end{course}
\end{document}
